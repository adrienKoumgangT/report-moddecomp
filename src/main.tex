%! Author = adrien koumgang tegantchouang
%! Date = 09/07/24

% Preamble
\documentclass[a4paper, 12pt]{report}
\usepackage[osf]{libertinus}
\pagestyle{plain}
\usepackage[top=2.5cm,bottom=2.5cm,left=4cm,right=2.5cm, centering]{geometry}
\linespread{1.5}
\usepackage[utf8]{inputenc} % codifica UTF-8
\usepackage{scrlayer-scrpage} % stili pagina per il frontespizio
\pagestyle{scrplain}
\usepackage{mathptmx} % font Times New Roman (simile)
\usepackage{graphicx, wrapfig}
\usepackage{lipsum}

\usepackage{url}
\usepackage{changepage}
\usepackage{multicol}
\usepackage{caption}
\usepackage{subcaption} % For the images' captions
\usepackage{amsmath}
\usepackage{amsthm} % Theorem styles
\usepackage{amssymb} % \mathbb and others
\usepackage{mathrsfs} % An alternative font for categories and sheaves.
\usepackage{cite} % For multiple citations
% \usepackage{caption}
% \usepackage{subcaption} % For the images' captions
\usepackage[section]{placeins} % To not break (sub)-images in two
\usepackage{booktabs}
\usepackage{float} % Always for images
\usepackage{soul} % For the \ul command, to use lists with less vertical spacing
\usepackage{enumerate} % For (i) list style
\usepackage{glossaries}
% \usepackage{minted}

\usepackage{blindtext}
\usepackage{hyperref}

%\usepackage{biblatex} % Imports biblatex package
%\addbibresource{mybibliography.bib} % Import the bibliography file

\usepackage{listings}
% \usepackage{listings-rust}
% \usepackage{rustlistings}
\usepackage{xcolor} % Optional: for colored code
\renewcommand{\lstlistingname}{Code}% Listing -> Codice\renewcommand{\lstlistingname}{Code}% Listing -> Codice

\usepackage[noend]{algpseudocode}
\algnewcommand\algorithmicforeach{\textbf{for each}}
\algdef{S}[FOR]{ForEach}[1]{\algorithmicforeach\ \#1 \algorithmicdo}

% \usepackage{color}
\usepackage{ragged2e}
\usepackage{algorithmicx}
\usepackage{algorithm}
% \usepackage{algorithm2e}
\usepackage{program}
% \usepackage{listings-rust}
\usepackage{mathtools}



% \usepackage{xcolor}
\definecolor{codegreen}{rgb}{0,0.6,0}
\definecolor{codegray}{rgb}{0.5,0.5,0.5}
\definecolor{codemauve}{rgb}{0.4,0.,0.4}
% \definecolor{codepurple}{rgb}{0.5,0.5,0.5}
% \definecolor{backcolour}{rgb}{0.95,0.95,0.92}

% Define a custom color
\definecolor{backcolour}{rgb}{0.95,0.95,0.92}
\definecolor{codegreen}{rgb}{0,0.6,0}
\definecolor{codekeywork3}{rgb}{0.82,0.56,0.43}
% Define a custom style
\lstdefinestyle{cpp}{
    language = C++,
    backgroundcolor=\color{backcolour},   % choose the background color
    basicstyle=\footnotesize,        % the size of the fonts that are used for the code
    breakatwhitespace=false,         % sets if automatic breaks should only happen at whitespace
    breaklines=true,                 % sets automatic line breaking
    captionpos=b,                    % sets the caption-position to bottom
    commentstyle=\color{codegreen},    % comment style
% deletekeywords={...},            % if you want to delete keywords from the given language
% escapeinside={\%*}{*)},          % if you want to add LaTeX within your code
    extendedchars=true,              % lets you use non-ASCII characters; for 8-bits encodings only, does not work with UTF-8
    firstnumber=1000,                % start line enumeration with line 1000
    frame=single,                    % adds a frame around the code
    keepspaces=true,                 % keeps spaces in text, useful for keeping indentation of code (possibly needs columns=flexible)
    keywordstyle=\color{blue},       % keyword style
    keywordstyle=[2]\color{blue},       % keyword style
    keywordstyle=[3]\color{codekeywork3},       % keyword style
    language=Octave,                 % the language of the code
% morekeywords={*},            % if you want to add more keywords to the set
    morekeywords=[2]{*, class, private, public},            % if you want to add more keywords to the set
    morekeywords=[3]{*, unsigned, int, bool, vector, set, tuple},            % if you want to add more keywords to the set
    numbers=left,                    % where to put the line-numbers; possible values are (none, left, right)
    numbersep=5pt,                   % how far the line-numbers are from the code
    numberstyle=\tiny\color{codegray}, % the style that is used for the line-numbers
    rulecolor=\color{black},         % if not set, the frame-color may be changed on line-breaks within not-black text (e.g. comments (green here))
    showspaces=false,                % show spaces everywhere adding particular underscores; it overrides 'showstringspaces'
    showstringspaces=false,          % underline spaces within strings only
    showtabs=false,                  % show tabs within strings adding particular underscores
    stepnumber=1,                    % the step between two line-numbers. If it's 1, each line will be numbered
    stringstyle=\color{codemauve}, % string literal style
    tabsize=4, % sets default tabsize to 2 spaces
    alsoother={\<},
    alsoother={\>},
    morecomment=[l]{//},
    morecomment=[s]{/*}{*/}
}

\lstdefinestyle{rust}{
    language = Rust,
    backgroundcolor=\color{backcolour},   % choose the background color
    basicstyle=\footnotesize,        % the size of the fonts that are used for the code
    breakatwhitespace=false,         % sets if automatic breaks should only happen at whitespace
    breaklines=true,                 % sets automatic line breaking
    captionpos=b,                    % sets the caption-position to bottom
    commentstyle=\color{codegreen},    % comment style
% deletekeywords={...},            % if you want to delete keywords from the given language
% escapeinside={\%*}{*)},          % if you want to add LaTeX within your code
    extendedchars=true,              % lets you use non-ASCII characters; for 8-bits encodings only, does not work with UTF-8
    firstnumber=1000,                % start line enumeration with line 1000
    frame=single,                    % adds a frame around the code
    keepspaces=true,                 % keeps spaces in text, useful for keeping indentation of code (possibly needs columns=flexible)
    keywordstyle=\color{blue},       % keyword style
    keywordstyle=[2]\color{blue},       % keyword style
    keywordstyle=[3]\color{codekeywork3},       % keyword style
    language=Octave,                 % the language of the code
% morekeywords={*},            % if you want to add more keywords to the set
    morekeywords=[2]{*, pub, struct, fn, impl, let, match, as},            % if you want to add more keywords to the set
    morekeywords=[3]{*, bool, char, f32, f64, i8, i16, i32, i64, isize, str, u8, u16, u32, u64, unit, usize, i128, u128},            % if you want to add more keywords to the set
    numbers=left,                    % where to put the line-numbers; possible values are (none, left, right)
    numbersep=5pt,                   % how far the line-numbers are from the code
    numberstyle=\tiny\color{codegray}, % the style that is used for the line-numbers
    rulecolor=\color{black},         % if not set, the frame-color may be changed on line-breaks within not-black text (e.g. comments (green here))
    showspaces=false,                % show spaces everywhere adding particular underscores; it overrides 'showstringspaces'
    showstringspaces=false,          % underline spaces within strings only
    showtabs=false,                  % show tabs within strings adding particular underscores
    stepnumber=1,                    % the step between two line-numbers. If it's 1, each line will be numbered
    stringstyle=\color{codemauve}, % string literal style
    tabsize=4, % sets default tabsize to 2 spaces
    alsoother={\<},
    alsoother={\>},
    morecomment=[l]{//},
    morecomment=[s]{/*}{*/}
}

\lstdefinestyle{python}{
    language=Python,                    % Set language to Python
    basicstyle=\ttfamily\footnotesize,   % Basic font style for the code
    keywordstyle=\color{blue}\bfseries,  % Style for keywords
    stringstyle=\color{red},             % Style for strings
    commentstyle=\color{gray}\itshape,   % Style for comments
    numberstyle=\tiny\color{gray},       % Style for line numbers
    numbers=left,                        % Show line numbers on the left
    stepnumber=1,                        % Line numbers every 1 line
    numbersep=8pt,                       % Space between line numbers and code
    tabsize=4,                           % Width of a tab
    showspaces=false,                    % Don't show spaces
    showstringspaces=false,              % Don't show spaces in strings
    showtabs=false,                      % Don't show tab characters
    frame=single,                        % Add a frame around the code
    rulecolor=\color{black},             % Frame color
    captionpos=b,                        % Position of caption (bottom)
    breaklines=true,                     % Automatically break long lines
    breakatwhitespace=true,              % Allow breaking at whitespace
    backgroundcolor=\color{white},       % Set background color
    morekeywords={self, None, True, False}, % Additional Python-specific keywords
    % escapeinside={(*@}{@*)},             % Escape inside code
}

% \lstdefinestyle{mystyle}{
%    backgroundcolor=\color{backcolour},
%    commentstyle=\color{codegreen},
%    keywordstyle=\color{magenta},
%    numberstyle=\tiny\color{codegray},
%    stringstyle=\color{codepurple},
%    basicstyle=\ttfamily\footnotesize,
%    breakatwhitespace=false,
%    breaklines=true,
%    captionpos=b,
%    keepspaces=true,
%    numbers=left,
%    numbersep=5pt,
%    showspaces=false,
%    showstringspaces=false,
%    showtabs=false,
%    tabsize=4
% }


\newtheorem{mydef}{Definition}
\newtheorem{mytheo}{Theorem}
\newtheorem{mylem}{Lemma}
\newtheorem{mypro}{Property}
\newtheorem{myex}{Example}
\newtheorem{mycor}{Corollary}
\newtheorem{myproof}{Proof idea}
\newtheorem{mypc}{Pseudocode}



\begin{document}
    \begin{titlepage} %crea l'enviroment
\begin{figure}[t] %inserisce le figure
    \centering\includegraphics[width=0.80\textwidth]{marchio_unipi_pant541}\label{fig:figure-first-page}
\end{figure}

\vspace{20mm}

\begin{Large}
 \begin{center}
	\textbf{Dipartimento di Informatica\\ Corso di Laurea Triennale in Informatica\\}
	\vspace{2cm}
    {\LARGE{TESI DI LAUREA}}\\
	\vspace{2cm}
	{\Large{Modular Decomposition of Graphs: Algorithmic Aspects and Implementation in Rust}}
\end{center}
\end{Large}

\vspace{15mm}

%minipage divide la pagina in due sezioni settabili
\begin{minipage}[t]{0.47\textwidth}
	{\large{\textbf{Relatore esterno:\\ Prof. Frédéric Peschanshi \\ Prof. Antoine Genitrini}}}

	\vspace{0.5cm}

	{\large{\textbf{Relatore interno: \\ Prof. Andrea Corradini}}}
\end{minipage}
\hfill
\begin{minipage}[t]{0.47\textwidth}\raggedleft
	{\large{\textbf{Candidato: \\ Adrien Koumgang Tegantchouang}}}
\end{minipage}

\vspace{20mm}

\centering{\large{\textbf ANNO ACCADEMICO 2023/2024 }}

\end{titlepage}


    \tableofcontents

% \listoffigures


    \thispagestyle{empty}

    \clearpage
    \setcounter{page}{1}


% Preambule (A Survey on Algorithmic Aspects of Modular Decomposition)
% \include{chapters/chapter_00}


% Introduction
    %! Author = adrienkoumgangtegantchouang
%! Date = 09/07/24

\chapter{Introduction}\label{ch:introduction}


\section{Overview}\label{sec:overview}

The field of graph theory is a cornerstone of computer science and discrete mathematics, offering a rich framework for modeling and analyzing complex networks.
Among the various techniques and algorithms developed within this domain, modular decomposition stands out due to its ability to simplify and break down graphs into smaller, more manageable components called modules.
This decomposition is pivotal for solving numerous combinatorial optimization problems, including those found in bioinformatics, social network analysis, and communication networks.

This thesis aims to delve into the modular decomposition of graphs, focusing on its theoretical foundations, algorithmic implementations, and practical performance evaluations.
By building upon the work of Eleni Pistiloglou, who explored the modular decomposition of 2-structures, this research seeks to extend her findings and improve upon them through the implementation of the modular decomposition algorithm in Rust, a programming language known for its performance and safety features.


\section{Objectives}\label{sec:objectives}

The main objectives of this thesis are as follows:

\begin{itemize}
    \item \textbf{Understanding Modular Decomposition:} To comprehensively understand the concept of modular decomposition, its definitions, properties, and applications in graph theory and beyond.
    \item \textbf{Extending Previous Work:} To build upon Eleni Pistiloglou's work on 2-structures, enhancing the understanding of modular decomposition in this context.
    \item \textbf{Algorithm Implementation:} To implement the modular decomposition algorithm in Rust and in C++.
    \item \textbf{Performance Comparison:} To rigorously evaluate and compare the performance of the Rust implementation against Python and SageMath, particularly focusing on execution time and scalability with different graph sizes.
\end{itemize}


\section{Structure of the Thesis}\label{sec:structure-of-the-thesis}

This thesis is structured to guide the reader through the comprehensive study and findings of this research.
The chapters are organized as follows:

\begin{itemize}
    \item \textbf{Chapter \ref{ch:modular-decomposition}: Modular Decomposition}: This chapter delves into the theoretical aspects of modular decomposition, defining key concepts such as modules, maximal modular partitions, and the application of these concepts to oriented graphs and 2-structures.
    \item \textbf{Chapter \ref{ch:implementation-of-modular-decomposition}: Implementation of modular decomposition}: This chapter presents the modular decomposition algorithm, detailing its steps, pseudocode, and a comprehensive example to illustrate the algorithm's application.
    \item \textbf{Chapter \ref{ch:previous-work-implementation-in-sagemath-and-python}: Previous Work: Implementation in SageMath and Python}: This chapter presents the work previously done, including the SageMath and Python implementations, with a particular focus on the Python version and its performance against SageMath.
    \item \textbf{Chapter \ref{ch:implementation-in-rust-and-cpp}: Implementation in Rust and C++}: This chapter discusses the implementation process in C++ and Rust, highlighting the transition from C++ to Rust and comparing the Rust implementation with the existing Python version.
    \item \textbf{Chapter \ref{ch:benchmarking}: Benchmarking}: This chapter outlines the methodology used for performance testing, presents the results, and provides an in-depth analysis of the performance of the different implementations.
\end{itemize}



% modular decomposition
    %! Author = adrienkoumgangtegantchouang
%! Date = 09/07/24

\chapter{Modular Decomposition}\label{ch:modular-decomposition}


Modular decomposition is a fundamental concept in graph theory and combinatorial optimization that simplifies the analysis and processing of graphs.
By breaking down a graph into smaller, more manageable subgraphs known as modules, modular decomposition enables more efficient algorithms for various graph-related problems.
This chapter provides an overview of modular decomposition, including its definitions, properties, applications, and examples.


\section{Definition of graph}\label{sec:definition-of-graph}

A graph is a mathematical structure used to model pairwise relations between objects. \cite{GT1,GT2}

\begin{mydef}
    A graph $G$ is defined as an ordered pair $G = (V, E)$, where
    \begin{itemize}
        \item $V$ is a finite set of vertices (or nodes).
        \item $E$ is a set of edges (or arcs), where each edge is a pair of vertices.
    \end{itemize}
\end{mydef}

Graphs can be broadly categorized into two types based on the nature of their edges: undirected graphs and directed graphs.

\subsection*{Undirected Graph}\label{subsec:undirected-graph}

An undirected graph is a type of graph in which the edges have no orientation.
The edges simply connect two vertices and do not have a direction associated with them.

An undirected graph is a graph $G = (V, E)$ where $E \subseteq P_2(V)$, i.e., $E$ is a set of subsets of nodes of cardinality 2.
Therefore, each edge $\{u, v\}$ is an unordered pair of vertices.

In undirected graphs, the edge $\{u, v\}$ is identical to the edge $\{v, u\}$, implying that the connection is bidirectional.

\begin{myex}[An undirected graph~\cite{mdwikipedia}]
    An undirected graph $G$ with vertices $V = \{v1, v2, v3, v4, v5, v6, v7, v8, v9, v10, v11\}$ and edges \\
    $E= \{\{v1, v2\}, \{v1, v3\}, \{v1, v4\}\}$
    $\cup \; \{\{v2, v4\}, \{v2, v5\}, \{v2, v6\}, \{v2, v7\}\}$ \\
    $\cup \; \{\{v3, v4\}, \{v3, v5\}, \{v3, v6\}, \{v3, v7\}\}$
    $\cup \; \{\{v4, v5\}, \{v4, v6\}, \{v4, v7\}\}$
    $\cup \; \{\{v5, v6\}, \{v5, v7\}\}$
    $\cup \; \{\{v6, v8\}, \{v6, v9\}, \{v6, v10\}, \{v6, v11\}\}$
    $\cup \; \{\{v7, v8\}, \{v7, v9\}, \{v7, v10\}, \{v7, v11\}\}$
    $\cup \; \{\{v8, v9\}, \{v8, v10\}, \{v8, v11\}\}$
    $\cup \; \{\{v9, v10\}, \{v9, v11\}\}$

    \begin{figure}[!h]
        \centering
        \includegraphics[width=0.40\textwidth]{images/graphs/undirected_graph_wikipedia}
        \caption{Example of Undirected Graph}
        \label{fig:example-undirected-graph}
    \end{figure}
\end{myex}

Undirected graphs are commonly used in scenarios where the relationships are mutual, such as social networks where friendships are bidirectional.

\subsection*{Directed Graph}\label{subsec:directed-graph}

A directed graph, or digraph, is a type of graph in which the edges have a direction, indicating a one-way relationship between the vertices.

A directed graph is a graph $G = (V, E)$ where $E \subseteq V \times V$, i.e., $E$ is a binary relation on $V$.
Therefore, each edge $(v, w)$ is an ordered pair of vertices.

In directed graphs, the edge $(u, v)$ is not identical to the edge $(v, u)$, implying that the connection is unidirectional.

\begin{myex}[A directed graph]
    A directed graph $G$ with vertices $V = \{1, 2, 3, 4, 5, 6, 7, 8\}$ and edges \\
    $E = \{(1, 2), (1, 3)\}$
    $\cup \; \{(2, 3)\}$
    $\cup \; \{(4 , 1), (4, 2), (4, 3), (4, 6)\}$
    $\cup \; \{(5 , 1), (5, 2), (5, 3), (5, 6)\}$
    $\cup \; \{(6, 4), (6, 5), (6, 7), (6, 8)\}$
    $\cup \; \{(7, 8)\}$
    $\cup \; \{(8, 7)\}$

    \begin{figure}[!h]
        \centering
        \includegraphics[width=0.40\textwidth]{images/graphs/digraph_ex1_without_label}
        \caption{Example of Directed Graph}
        \label{fig:example-directed-graph}
    \end{figure}
\end{myex}

Directed graphs are used where relationships have a direction, such as web page links (where one page links to another) or task scheduling (where one task must precede another).


\section{Definition of Modular Decomposition}\label{sec:definition-of-modular-decomposition}

\subsection*{Module}\label{subsec:module}

Modular decomposition involves partitioning the vertex set $V$ into subsets, or modules, that have specific properties relative to the rest of the graph.

A \textbf{module}~\cite{mdwikipedia} of a graph is a generalization of a connected component~\cite{componentwikipedia}.
In graph theory, a \textbf{component} of an undirected graph is a connected subgraph that is not part of any larger connected subgraph.
The components of any graph partition its vertices into disjoint sets, and are the induced subgraphs of those sets.
A graph that is itself connected has exactly one component, consisting of the whole graph.
Components are sometimes called \textbf{connected components}.

\begin{figure}[!h]
    \centering
    \includegraphics[width=0.40\textwidth]{images/graphs/Pseudoforest}
    \caption{Example of Modules (A graph with three components) \cite{componentwikipedia}}
    \label{fig:example-modules}
\end{figure}

A connected component has the property that it is a set $M$ of vertices such that every member of $M$ is a non-neighbor of every not vertex in $M$.
(It is a union of connected components if and only if it has this property.)
More generally, $M$ is a module if, for each vertex $m \notin M$, either every member of $M$ is a non-neighbor of $m$ or every member of $M$ is a neighbor of $m$.

Equivalently, $M$ is a module if all members of $M$ have the same set of neighbors among vertices not in $M$.

Another definition of components involves the equivalence classes of an equivalence relation defined on the graph's vertices.
In an undirected graph, a vertex $v$ is reachable from a vertex $u$, if there is a path from $u$ to $v$, or equivalently a walk (a path allowing repeated vertices and edges).
Reachability is an equivalence relation, since:
\begin{itemize}
    \item It is \textbf{reflexive}: There is a trivial path of length zero from any vertex to itself.
    \item It is \textbf{symmetric}: If there is a path from $u$ to $v$, the same edges in the reserve order from a path from $v$ to $u$.
    \item It is \textbf{transitive}: If there is a path from $u$ to $v$ and a path from $v$ to $w$, the two paths may be concatenated together to form a walk from $u$ to $w$.
\end{itemize}

\begin{mydef}
(Module)
    A module $M \subseteq V$ of a graph $G$ is a subset of vertices such that every vertex in $M$ has the same set of neighbors outside $M$.
    Formally, for any $u, v \in M$ and any $w \in V - M$, $w$ is adjacent to $u$ if and only if $w$ is adjacent to $v$.
\end{mydef}

This definition ensures that within a module, the vertices are indistinguishable based on their connections to the rest of the graph.
Modules can vary in size and structure, from single vertices to large subgraphs encompassing a significant portion of the original graph.

Contrary to the connected components, the modules of a graph are the same as the modules of its complement, and modules can be nested: one module can be a proper subset of another.
Note that the set $V$ of vertices of a graph is a module, as are its one-element subsets and the empty set; these are called the \textbf{trivial modules}.
A graph may or may not have other modules.
A graph is called \textbf{prime} if all of its modules are trivial.

\begin{mydef}[Prime module]
    A prime module is a module where the subgraph induced by the vertices of the module does not admit any non-trivial decomposition into smaller modules.
\end{mydef}

In the mathematical field of graph theory, the complement~\cite{complementgraphwikipedia} or inverse of a graph $G$ is a graph $H$ on the same vertices such that two distinct vertices of $H$ are adjacent if and only if they are not adjacent in $G$.
That is, to generate the complement of a graph, one fills in all the missing edges required to form a complete graph, and removes all the edges that were previously there.
The complement is not the set complement of the graph; only the edges are complemented.

\begin{figure}[!h]
    \centering
    \includegraphics[width=0.40\textwidth]{images/graphs/Petersen_graph_complement}
    \caption{The Petersen graph (on the left) and its complement graph (on the right) \cite{complementgraphwikipedia}}
    \label{fig:the}
\end{figure}


\subsection*{Modular Partition}\label{subsec:modular-partition}

In graph theory, partitioning a graph into meaningful substructures is a common strategy for simplifying complex problems.
A modular partition specifically refers to dividing the vertices of a graph into disjoint subsets, known as modules, where each subset satisfies certain uniformity conditions relative to the rest of the graph.
This concept is crucial for understanding the deeper hierarchical structure of graphs and forms the foundation for various graph decomposition techniques.

\begin{mydef}
    A modular partition of a graph $G = (V, E)$ is a partition of the vertex set $V$ into disjoint subsets $M_1$, $M_2$, \ldots, $M_k$ such that each subset $M_i$ (for $i = 1, 2, \ldots, k$) is a module.
\end{mydef}

\subsubsection*{Properties of Modular Partitions}

Modular partitions possess several important properties that make them useful for graph analysis:

\begin{enumerate}
    \item \textbf{Uniform Neighborhood:} Within a module, all vertices share the same neighbors outside the module.
    This property allows for simplification of the graph by treating the entire module as a single unit when considering its external connections.
    \item \textbf{Independence from Internal Structure:} The internal structure of a module does not affect its status as a module.
    This means that the relationships between vertices within a module are irrelevant to the modular partitioning, focusing solely on external connections.
    \item \textbf{Hierarchical Organization:} Modular partitions naturally lead to a hierarchical organization of the graph.
    By recursively partitioning modules into smaller modules, one can build a modular decomposition tree that captures the multi-level structure of the graph.
\end{enumerate}

To illustrate the concept of modular partitions, consider the following examples:

\begin{myex}[Simple Undirected Graph]
    Consider an undirected graph $G$ from Figure~\ref{fig:example-undirected-graph}.

    In this graph, the sets $\{v2, v3, v4\}$ and $\{v6, v7\}$ each form modules.
    The non trivial modules are: $\{v2, v3\}, \{v2, v3, v4\}, \{v6, v7\}, \{v10, v11\}, \{v8, v9, v10, v11\}$.

    \begin{figure}[!h]
        \centering
        \includegraphics[width=0.40\textwidth]{images/graphs/undirected_graph_wikipedia_module}
        \caption{Example of Module for Simple Undirected Graph}
        \label{fig:example-undirected-graph-module}
    \end{figure}

    Possible modular partition are:
    \begin{itemize}
        \item $\{\{v1\}, \{v2, v3, v4\}, \{v5\}, \{v6, v7\}, \{v8, v9, v10, v11\}\}$
        \item $\{\{v1\}, \{v2, v3\}, \{v4\}, \{v5\}, \{v6, v7\}, \{v8\}, \{v9\}, \{v10, v11\}\}$
    \end{itemize}
    The last one is the maximal modular partition.
\end{myex}

\begin{myex}[Directed Graph]
    Consider a directed graph $G$ from Figure~\ref{fig:example-directed-graph}.

    In this graph, the sets $\{2, 3\}$, $\{4, 5\}$, and $\{6, 7, 8\}$ each form modules.
    Thus, a possible modular partition of $G$ is $\{\{1\}, \{2, 3\}, \{4, 5\}, \{6, 7, 8\}\}$.
\end{myex}

\subsection*{Maximal Modular Partition}\label{subsec:maximal-modular-partition}

The maximal modular partition of a graph is unique and consists of the largest possible prime modules.
Prime modules are those that cannot be further decomposed into smaller non-trivial modules.

\begin{mydef}
(Maximal modular partition)
    A maximal modular partition is a partition of a graph into modules, where the modules are maximal in the sense that they cannot be further partitioned into smaller modules.
\end{mydef}


\section{Application to 2-Structures}\label{sec:application-to-2-structures}

Modular decomposition extends naturally to 2-structures, which are complete directed graphs with arcs colored from a set of $k$ colors.
The additional complexity of orientation and coloring requires modified definitions and algorithms for modular decomposition.

A 2-structure is a generalization of a graph where edges (or arcs) between vertices are assigned different colors or types, capturing more complex relationships between vertices.
This concept is particularly useful in scenarios where the simple binary relationship (edge or no edge) in standard graphs is insufficient to model the nuances of the interactions between entitites.

\begin{mydef}
    A 2-structure $G$ is defined as a pair $G = (V, A)$, where
    \begin{itemize}
        \item $V$ is a finite set of vertices.
        \item $A$ is a set of arcs, where each arc is a triple $(u, v, c)$ consisting of an ordered pair of vertices $u$ and $v$, and a color or type $c$.
    \end{itemize}
\end{mydef}

In a 2-structure, the arc $(u, v, c)$ indicates a directed relationship from vertex $u$ to vertex $v$ with a specific color or type $c$.

Here are the properties of the 2-structures:

\begin{enumerate}
    \item \textbf{Directed and Colored Arcs:} Unlike standard graphs where edges are typically uncolored, 2-structures have arcs that are both directed and colored.
    This allows for a richer representation of relationships.
    \item \textbf{Complex Relationships:} The use of colors or types for arcs allows 2-structures to model complex relationships between vertices, where the nature of the relationship is significant.
    \item \textbf{Generalization:} 2-structures generalize both undirected and directed graphs.
    An undirected graph can be seen as a 2-structure where each edge is represented by two directed arcs of the same color (one in each direction).
    A directed graph is a 2-structure where arcs are colored uniformly.
\end{enumerate}

\begin{myex}[2-Structure]
    Consider a 2-structure $G$ with vertices $V = \{u, v, w, x\}$ and arcs $A = \{(u, v, 1), (u, w, 1), (v, w, 2), (w, x, 1)\}$, where the third element in each triple represents the color of the arc.

    \begin{figure}[!h]
        \centering
        \includegraphics[width=0.40\textwidth]{images/graphs/2_structure_graph_example}
        \caption{Example of 2 Structure Graph}
        \label{fig:2-structure-graph-example-simple}
    \end{figure}

    In this 2-structure:
    \begin{itemize}
        \item The set of vertices $V$ is $\{u, v, w, x\}$
        \item The set of arcs $A$ includes:
        \begin{itemize}
            \item $(u, v, 1)$ indicating a type-1 arc from $u$ to $v$.
            \item $(u, w, 1)$ indicating a type-1 arc from $u$ to $w$.
            \item $(v, w, 2)$ indicating a type-2 arc from $v$ to $w$.
            \item $(w, x, 1)$ indicating a type-1 arc from $w$ to $x$.
        \end{itemize}
    \end{itemize}
\end{myex}

Another example of 2-structure from directed graph is Figure~\ref{fig:example-directed-graph}.


The definition of modules in 2-structures incorporates both the direction and color of arcs.

\begin{mydef}
    A module $M$ in a 2-structure $G$ is a subset of vertices such that for any $u, v \in M$ and any $w \in V - M$, the arcs $(w, u, c)$ and $(w, v, c)$ have the same color and direction, and the arcs $(u, w, c)$ and $(v, w, c)$ have the same color and direction.
\end{mydef}


\section{Examples of Modular Decomposition}\label{sec:examples-of-modular-decomposition}

To illustrate the concept of modular decomposition, consider the following examples.

% \subsection*{Example 1: Simple Undirected Graph}\label{subsec:example-1:-simple-undirected-graph}

\begin{myex}[Simple Undirected Graph]
    \label{ex:simple-undirected-graph}

Consider an undirected graph $G$ from Figure~\ref{fig:example-undirected-graph} and Figure~\ref{fig:example-undirected-graph-module}, the non-trivial modules of this graph are:

\begin{itemize}
    \item $\{v2, v3, v4\}$.
    \item $\{v6, v7\}$.
    \item $\{v8, v9, v10, v11\}$.
\end{itemize}

The maximal modular partition of $G$ is $\{\{v1\}, \{v2, v3\}, \{v4\}, \{v5\}, \{v6, v7\},$ \\ $\{v8\}, \{v9\}, \{v10, v11\}\}$.
\end{myex}

\begin{myex}[Directed Graph (2-structure)]
    Consider a directed graph $G$ from Figure~\ref{fig:example-directed-graph}.

    The non-trivial modules of this graph are $\{2, 3\}$, $\{4, 5\}$ and $\{7, 8\}$

    The maximal modular partition of $G$ is $\{\{1\}, \{2, 3\}, \{4, 5\}, \{6\}, \{7, 8\}\}$.
\end{myex}


\section{Modular Decomposition Tree}\label{sec:modular-decomposition-tree}

A modular decomposition tree is a tree structure that represents the hierarchical modular decomposition of a graph.

Structure of the Modular Decomposition Tree:
\begin{itemize}
    \item \textbf{Nodes:} Each node in the tree represents a module of the graph.
            \begin{itemize}
                \item \textbf{Leaf Nodes:} These correspond to individual vertices of the graph.
                \item \textbf{Internal Nodes:} These represent a module and are labeled with one of three types:
                        \begin{itemize}
                            \item \textbf{Parallel Node (P-node)}: Represents a module where the directed sub-modules form a set of disjoint independent sets (no edges between vertices within the module).
                            \item \textbf{Series Node (S-node)}: Represents a module where the directed sub-module form a clique (complete subgraph).
                            \item \textbf{Prime Node}: Represents a module where the module cannot be decomposed further into either parallel or series modules.
                        \end{itemize}
            \end{itemize}
    \item \textbf{Root:} The root node of the tree represents the entire graph.
\end{itemize}


\begin{Example}
    Consider the graph $G$ from Figure~\ref{fig:example-undirected-graph} with maximal modular partition \\ $\{\{v1\}, \{v2, v3, v4\}, \{v5\}, \{v6, v7\}, \{v8, v9\}, \{v10, v11\}\}$ (Figure~\ref{fig:example-undirected-graph-module}).
    The modular decomposition tree is constructed as follows:

    \begin{figure}[!h]
        \centering
        \includegraphics[width=0.40\textwidth]{images/graphs/undirected_graph_wikipedia_modular_decomposition}
        \caption{Example of Modular Decomposition Tree}
        \label{fig:example-undirected-graph-modular-decomposition-tree}
    \end{figure}
\end{Example}


\section{Applications of Modular Decomposition}\label{sec:applications-of-modular-decomposition}

Modular decomposition has numerous applications in various fields, including:

\subsection*{Bioinformatics}\label{subsec:bioinformatics}

In bioinformatics, modular decomposition is used to analyze and understand the structure of biological networks~\cite{MDPPIN}, such as protein-protein interaction networks and gene regulatory networks.
By decomposing these networks into modules, researchers can identify functional units and study their interactions.

\subsection*{Social Network Analysis}\label{subsec:social-network-analysis}

In social network analysis, modular decomposition helps identify communities or groups within a social network~\cite{NCCD}.
By understanding the modular structure, analysts can uncover patterns of interaction, influence, and information flow within the network.

\subsection*{Communication Networks}\label{subsec:communication-networks}

In communication networks, modular decomposition is used to optimize network design and routing~\cite{MANTP}.
By decomposing the network into modules, network engineers can develop efficient routing algorithms and improve the overall performance and reliability of the network.

% \section*{Conclusion}\label{sec:conclusion}

\hspace{4cm}

Modular decomposition is a powerful tool in graph theory that simplifies the analysis and processing of complex graphs.
By breaking down a graph into smaller modules, it enables more efficient algorithms and provides valuable insights into the structure and properties of the graph.
This chapter has provided a comprehensive overview of modular decomposition, including its definitions, properties, applications, and examples.
The subsequent chapters will delve into the algorithmic implementation of modular decomposition and its performance evaluation.



% algorithms of Eh....
    %! Author = adrien koumgang tegantchouang
%! Date = 09/07/24

\chapter{Implementation of modular decomposition}\label{ch:implementation-of-modular-decomposition}

Modular decomposition is a powerful technique used to simplify and analyze the structure of graphs by breaking them down into modules.
One of the notable algorithms for achieving this is the algorithm developed by Ehrenfeucht et al.
This section delves into the details of the algorithm, its steps, and its application in graph theory.

\section{Modular Partition Algorithm}\label{sec:modular-partition-algorithm}

In his book~\cite{SAMD}, Ehrenf gives us an algorithm for performing modular decomposition on a graph $G = (V, E)$ :

\begin{algorithm}[H]
    \caption{Modular Partition}
    \KwIn{A partition $P$ of the vertex set $V$ of a graph $G$}
    \KwOut{The coarsest modular partition $Q$ smaller than $P$}
    \SetKwBlock{Begin}{begin}{end}
    \Begin{
        Let $Z$ be the largest part of $P$\;
        $Q \gets P$; $K \gets \{Z\}$; $L \gets \{X \mid X \neq Z, X \in P\}$\;
        \While{$L \cup K \neq \varnothing$}{
            \eIf{there exists $X \in L$}{
                $S \gets X$ and $L \gets L \setminus \{X\}$\;
            }{
                Let $X$ be the first part $K$ and $x$ arbitrarily selected in $X$\;
                $S \gets \{x\}$ and $K \gets K \setminus \{X\}$\;
            }
            \ForEach{vertex $x \in S$}{
                \ForEach{part $Y \neq X$ such that $N(x) \perp Y$}{
                    Replace in $Q$, $Y$ by $Y_1 = Y \cap N(x)$ and $Y_2 = Y \setminus N(x)$\;
                    Let $Y_{\min}$ (resp. $Y_{\max}$) be the smallest part (resp. largest) among $Y_1$ and $Y_2$\;
                    \eIf{$Y \in L$}{
                        $L \gets L \cup \{Y_{\min}, Y_{\max}\} \setminus \{Y\}$\;
                    }{
                        $L \gets L \cup \{Y_{\min}\}$\;
                        \If{$Y \in K$}{
                            Replace $Y$ by $Y_{\max}$ in $K$\;
                        }{
                            Add $Y_{\max}$ at the end of $K$\;
                        }
                    }
                }
            }
        }
    }
\end{algorithm}\label{alg:modular-partition-algorithm}

\begin{mytheo}
    Let $P$ be a partition of the vertices of a graph $G = (V, E)$.
    Algorithm~\ref{alg:modular-partition-algorithm} computes the coarsest modular partition for $G$ and $P$ in time $O(n + \log{n})$.
\end{mytheo}

The correctness of the algorithm follow from the next three invariant properties.
The first invariant shows that a module contains in some part of the given partition cannot be split, while the third one guarantees that the algorithm outputs a modular partition.

\begin{enumerate}
    \item If $M$ is a module of G contained in a part $X \in P$, then there exist a part $Y$ of the current partition containing $M$.
    \item If $L = \emptyset$, then the first part $Y$ of $K$is a module.
    \item If the current partition contains a part $X$ is not a module, the the exists $Y \in L \cup K$ different from $X$ and containing a splitter $y$ for $X$
\end{enumerate}

\textit{Complexity issues}: The main while loop manages a set $S$ of vertices who neighbourhoods have to be used to refine the current partition.
The set $S$ is computed from the lists $L$ and $K$.
Since the current part containing a given vertex can be added to $L$, only if its size is smaller than half of the size of the former part containing $x$, the neighbourhood of each vertex $x $is guaranteed to be visited at most $\log{(\mid V \mid)}$ times by the algorithm.
Furthermore, when a vertex $x$ of a part $X$ extracted from $K$ is used, neither $x$ nor none of the vertices of $X$ is used again.
This yields to a $O\left(\sum_{x \in V} \log{(\mid V \mid)}.\mid N (x) \mid\right)$ complexity, as claimed.


% \section{Recursive computation of the modular decomposition tree}\label{sec:recursive-computation-of-the-modular-decomposition-tree}

\begin{mydef}
    Let $v$ be an arbitrary vertex of a graph $G = (V, E)$.
    The v-modular partition is the following modular partition: \\
    $M(G, v) = \{v\} \cup \{M \mid M \text{ is a maximal module not containing } v\}$
\end{mydef}

The neighbourhood of a vertex $x$ in a graph $G = (V, E)$ is denoted $N_G(x)$ and its non-neighbourhood $\bar{N}_G(x)$ (subscript $G$ will be omitted when the context is clear).
The complementary graph of a graph $G$ is denoted by $G$.
Given a subset of vertices $X \subset V$ , $G[X]$ is the subgraph induced by $X$ (any edge in $G$ between two vertices in $X$ belongs to $G[X]$).

\begin{mylem}
    The partition $M(G, v)$ is the coarsest modular partition for $G$ and $P = \{N(v), v, \bar{N}(v)\}$ and can be computed in time $O(n + m \log n)$.
\end{mylem}

\begin{figure}[!h]
    \centering
    \includegraphics[width=0.80\textwidth]{images/graphs/modular-decomposition-tree}
    \caption{A modular decomposition tree}
    \label{fig:example-modular-decomposition-tree}
\end{figure}

\begin{algorithm}[H]
    \caption{Ehrenfeucht et al. \cite{SAMD}\cite{PTDMD}} % Reference citation in the caption
    \KwIn{An arbitrary vertex $v$ of $G = (V, E)$, $T = \text{spine}(G, v)$ and $\{T_X = MD(G[X]) \mid X \in \mathcal{M}(G,v)\}$}
    \KwOut{The modular decomposition tree $MD(G)$}
    \SetKwBlock{Begin}{begin}{end}
    \Begin{
        \ForEach{leaf $X$ of $T$}{
            Let $T_X = MD(G[X])$ and $p(X)$ be $X$'s father in $T$\;
            Replace $X$ by $T_X$ in $T$\;
            \If{the root $r(T_X)$ and $p(X)$ are both parallel or series}{
                Remove $r(T_X)$ and connect the children of $r(T_X)$ to $p(X)$\;
            }
        }
    }\label{alg:algorithm-Ehrenfeucht-et-al}
\end{algorithm}


\section{Advantages of Ehrenfeucht's Algorithm}\label{sec:advantages-of-ehrenfeucht's-algorithm}

\begin{itemize}
    \item \textbf{Efficiency:} The algorithm operates in $O(n + m\log{n})$ time, making it feasible for large graphs.
    \item \textbf{Simplicity:} The divide-and-conquer approach simplifies the process of identifying modules and constructing the decomposition tree.
    \item \textbf{Practicality:} The algorithm is applicable to both undirected graphs and 2-structures.
\end{itemize}


\hspace{4cm}

The next chapter will discuss an implementation of the algorithm in Python made by another Sorbonne University student and its performance compared with another implementation made in SageMath.


% Implementation in SageMath and Python
    %! Author = adrien koumgang tegantchouang
%! Date = 09/07/24

\chapter{Previous Work (Implementation in SageMath and Python)}\label{ch:previous-work-(implementation-in-sagemath-and-python)}


Eleni Pistiloglou, in her ``Rapport de projet'' for the Master 1 Informatique, parcours STL, has made significant contributions to modular decomposition implementation by implementing modular decomposition algorithms in Python.
Its implementation follows that of SageMath and its role is to improve the performance of the SageMath implementation.
This section provides an overview of her work, highlighting the methodologies and results of her implementations, which later served as a foundation for my work in converting the Python implementation to Rust.

\section{Overview of Eleni Pistiloglou's Project}\label{sec:overview-of-eleni-pistiloglou's-project}

Eleni Pistiloglou's project focuses on the modular decomposition of 2-structures, specifically directed graphs with colored arcs.
Her work involves:
\begin{itemize}
    \item Defining key concepts such as modules and maximal modular partitions.
    \item Implementing modular decomposition algorithms.
    \item Evaluating and comparing the performance of these implementations in Python.
\end{itemize}


To make his Python implementation, Eleni Pistiloglou first converted Ehrenfeucht's original algorithm into a pseudo-algorithm with the same characteristics and performance.
Here's the algorithm:

\begin{algorithm}
    \begin{algorithmic}
        \Function{modular\_decomposition}{$v$, $E$}
            \State Create a node $t$ in the decomposition tree
            \State \Call{decompose}{$t, V$}
            \State \Return $t$
        \EndFunction
    \end{algorithmic}\label{alg:modular-decomposition}
\end{algorithm}

\begin{algorithm}
    \begin{algorithmic}
        \Function{decompose}{$t$, $V$}
            \If{$\|V\| = 1$}
                \State $t = V$
            \Else
                \State $u \coloneqq t$
                \State Choose $v \in V$ at random
                \State $P = $ \Call{partition}{$V - {v},\, v$}
                \State Create the $G$, $G'$ and $G''$ graphs from $P$
                \While{$G'' != \emptyset$}
                    \State Remove the wells $p_1, \dots, p_n = Y$ in $G''$
                    \State \Call{treat}{$u, Y$}
                    \State Create an empty son $u'$ from $u$
                    \State $u \coloneqq u'$
                \EndWhile
                \State $u = \{v\}$
            \EndIf
        \EndFunction
    \end{algorithmic}\label{alg:decompose}
\end{algorithm}


\begin{algorithm}
    \begin{algorithmic}
        \Function{treat}{$u, Y$}
            \If{$|Y| = 1$ \textbf{and} $|p\_1| > 1$}
                \State // If there is only one well that contains more than a module
                \State \Call{label}{u} $=$ 'PRIME'
            \ElsIf{The color of a $V$ to $F$ arc is the same of $F$ to $v$ arc}
            \State \Call{label}{$u$} $=$ 'COMPLETE'
            \State \Call{color}{$u$} $=$ \Call{color}{$v, F$}
            \Else
                \State \Call{label}{$u$} $= 'LINEAR'$
                \State \Call{color}{$u$} $[$ \Call{color}{$v, F_1$}, \Call{color}{$F_1, v$} $]$
                \State // F1 is a module of F
            \EndIf

            \For{any module $M \in \bigcup_{p_i}$}
                \State $t' =$ \Call{modular\_decomposition}{$M$}
                \State // fusion
            \EndFor

            \If{$u$ and $t'$ are complete and the same color \textbf{or} linear and the same color}
                \State Add the sons of $t'$ to the sons of $u$
            \Else
                \State Add to the sons of $u$
            \EndIf
        \EndFunction
    \end{algorithmic}\label{alg:treat}
\end{algorithm}


\section{Implementation in Python}\label{sec:implementation-in-python}

Python, a high-level programming language known for its simplicity and versatility, was also used by Pistiloglou to implement the modular decomposition algorithm.

Firstly, it refers to the fact that the initial version of the code defines a TwoStructure class to represent a 2-structure that uses a dictionary to store arcs, whose key is the source index and whose value is a dictionary containing the destinations associated with the colour.
So the call to the function that decides whether a node distinguishes two others requires access to the values of both dictionaries.
Similarly, for the function that returns the colour of an arc.
The theoretical complexity of a dictionary search is $O(1)$ thanks to hashing, but implementing dictionary operations in Python is very time-consuming.
So her first attempt was to determine the most suitable data structures for representing the arcs and colours of a 2-structure and to design a new implementation for the parts that were dictionary-based.

The second version proposes an improved implementation of these functions that is possible using only sets or lists.
Sets are preferred over lists because their elements can be retrieved in $O(1)$ time thanks to hashing, unlike a search in a list, which is performed in $O(n)$ time.
Since the number of colours in a 2-structure is much smaller than the number of its arcs, a storage system that associates colours with source-destination pairs would improve the speed of calculation, which would be carried out in $O(K) \cdot O(1)$ time.
A bucket-sort solution stores each arc in an arc set that contains only arcs of the same colour.
The sets are stored in a list whose index corresponds to the colour of the arcs contained in that set.
The code for these two versions is available in her report.
She made other significant improvements to almost 50\% of the initial functions to adapt them to the changing data structures and to improve their completeness.
As a result, when tested on randomly structured graphs with 1000 nodes and 100 arcs, the total execution time for decomposing a directed graph was reduced to 0.152s in the second version, compared with 5.448s before the improvements.
Performance was also compared with SageMath on undirected graphs.
However, SageMath performed significantly better, with a latency twice as high as SageMath on graphs with 100 arcs.
For much larger graphs, such as 1000 arcs, we obtain an execution time of around 5 minutes with Python and 0.7 seconds with SageMath.
All these results can be found in her report.




% Implementation in Rust and C++
    %! Author = adrien koumgang tegantchouang
%! Date = 09/07/24


\chapter{Implementation in Rust and C++}\label{ch:implementation-in-rust-and-cpp}

The implementation of the modular decomposition algorithm in Rust was motivated by the need to overcome performance limitations observed in previous implementation in Python.
The primary motivation for implementing the modular decomposition algorithm in Rust was to achieve better performance and scalability for large and complex graphs.
As described on the official Rust website, we have that Rust is blazingly fast and memory-efficient: with no runtime or garbage collector, it can power performance-critical services, run on embedded devices, and easily integrate with other languages.
Rust's rich type system and ownership model guarantee memory-safety and thread-safety -- enabling you to eliminate many classes of buges at compile-time.
In brief, Rust is a programming language that adopts a very specific programming philosophy, mainly thanks to its ownership principle, which ensures memory safety without needing a garbage collector.
Ownership governs how memory is managed and how different parts of a program can access and modify data.

the key principles of ownership in Rust are~\cite{rust}:
\begin{itemize}
    \item \textbf{Ownership Rules}
    \begin{itemize}
        \item Each value in Rust has a variable that's called its owner.
        \item There can only be one owner at a time.
        \item When the owner goes out of scope, the value will be dropped (freed).
    \end{itemize}
    \item \textbf{Borrowing and References}
    \begin{itemize}
        \item He can borrow a value by creating a reference to it using `\&'.
        \item Borrowing can be either immutable (`\&T') or mutable (`\&mut T').
        \item Multiple immutable references are allowed, but only one mutable reference is allowed at a time.
    \end{itemize}
    \item \textbf{Move Semantics}
    \begin{itemize}
        \item When a value is assigned to another variable or passed to a function, the ownership of the value moves to the new variable or function parameter.
        \item After the move, the original variable is no longer accessible.
    \end{itemize}
    \item \textbf{Copy Trait}
    \begin{itemize}
        \item For types that implement the `Copy' trait (like integers and other simple types), a copy of the value is made rather than moving ownership.
    \end{itemize}
    \item \textbf{Lifetimes}
    \begin{itemize}
        \item Lifetimes are a way to ensure that references are valid as long as they are needed.
        \item They help the compiler reason about how long references should be valid and prevent dangling references.
    \end{itemize}
\end{itemize}

Coming as I do from programming languages such as C and Java, where you could have several pointers to a memory space in modification mode, where you could pass a pointer to a variable while still retaining control over it in write mode, Rust came to me as a real challenge.
The level of challenge was even greater because the code to be translated into Rust is written in Python~\cite{pythoncode}, which has the merit of being very flexible in terms of typing, variable access and memory areas~\cite{python}.
Many of the functions to be translated, especially the main one, were recursive, and some of them contained cycles that defined variables with a fairly complex life cycle.
As a result, I quickly ran into difficulties writing the code in Rust.
To overcome this, I turned to a programming language that has strongly influenced Rust: C++~\cite{cpp}.

This chapter provides an in-depth discussion of the methodologies, challenges and results associated with the implementation in Rust and the reasons why I decided to implement also in C++.


\section{Implementation in Rust: Graph Representation, Algorithm Design and Optimization Techniques}\label{sec:implementation-in-rust:-graph-representation-algorithm-design-and-optimization-techniques}

This section delves into the detailed aspects of implementing the algorithm in Rust, covering the graph representation, algorithm design, and optimization techniques.

\subsection{Graph Representation}\label{subsec:graph-representation}

For the representation of the various graphs used during the execution of the modular decomposition algorithm, such as the 2-structures, I followed the implementation made by Eleni Pistiloglou in her second implementation in Python, which uses mainly sets instead of lists to optimise access to both nodes and arcs.

\subsubsection{2-structure}

we therefore have the following definition for the representation of a 2-structure:
\begin{lstlisting}[language=Rust, style=rust, caption={Defining the 2-structure with Rust}, label={lst:rust-define-twostructure}, firstnumber=1]
    #[derive(Debug)]
    #[derive(Clone)]
    pub struct TwoStructure {
        /// the graph of a two structure
        pub nodes: HashSet<u64>,
        /// list of sets, colors[i] contains the set of edges of color i
        pub colors: Vec<HashSet<(u64, u64)>>,
        /// only for quotient graph, contains the nodes of the graph if they are modules
        pub modules: Vec<HashSet<u64>>,
    }
\end{lstlisting}
where we have:
\begin{itemize}
    \item \textbf{nodes}: the graph of a two structure
    \item \textbf{colors}: list of sets, colors[i] contains the set of edges of color i
    \item \textbf{modules}: only for quotient graph, contains the nodes of the graph if they are modules
\end{itemize}

\subsubsection{SCC}

SCC, for Strongly Connected Components, is a structure that takes a graph as input and determines the `strongly connected components' of this graph using its two visit and compute functions.
For doing this, we use `Tarjan's strongly connected components algorithm'~\cite{sccalgowikipedia}.
Tarjan's strongly connected components algorithm is an algorithm in graph theory for finding the strongly connected components (SCCs) of a directed graph.
It runs in linear time, matching the time bound for alternative methods including Kosaraju's algorithm and the path-based strong component algorithm.
The algorithm is named for its inventor, Robert Tarjan.

\begin{lstlisting}[language=Rust, style=rust, caption={Defining the SCC with Rust}, label={lst:rust-define-scc}, firstnumber=1]
    pub struct SCC<'a> {
        pub graph: Box<dyn GraphTrait + 'a>,
        pub index: u64,
        pub components: Vec<HashSet<u64>>,
        pub comp_index: u64,
        pub dfs_stk: Vec<SCCNode>,
        pub visited: HashMap<u64, SCCNode>,
    }
\end{lstlisting}

\begin{itemize}
    \item \textbf{graph}: This attribute holds the graph on which Tarjan's algorithm will be executed.
    It provides the vertices and the adjacency list needed to traverse the graph.
    \item \textbf{index}: This attribute is used to assign a unique discovery index to each vertex when it is first visited during the depth-first search (`DFS').
    It is incremented with each new vertex visit.
    \item \textbf{components}: This attribute stores the strongly connected components (`SCCs') identified in the graph.
    Each SCC is represented as a set of vertices, and all such sets are collected in this list.
    \item \textbf{comp\_index}:
    \item \textbf{dfs\_stk}: This attribute acts as a stack to manage the depth-first search traversal.
    It helps in tracking the current path and managing backtracking, which is essential for updating the `lowlink' values and identifying SCCs.
    \item \textbf{visited}: This attribute keeps track of all the vertices that have been visited and their associated SCCNode instances.
    It helps in quickly checking whether a vertex has already been visited and accessing its SCCNode information.
\end{itemize}

where GraphTrait is:

\begin{lstlisting}[language=Rust, style=rust, caption={Defining GraphTrait}, label={lst:rust-define-graphtrait}, firstnumber=1]
    pub trait GraphTrait {
        fn vertices(&self) -> HashSet<u64>;
        fn successors(&mut self, v: u64) -> &HashSet<u64>;
    }
\end{lstlisting}

\subsubsection{SCCNode}

\begin{lstlisting}[language=Rust, style=rust, caption={Defining the SCCNode with Rust}, label={lst:rust-define-sccnode}, firstnumber=1]
    #[derive(Clone, Copy)]
    pub struct SCCNode {
        // original vertex ... must be hashable and unique (among vertices)
        pub vertex: u64,
        pub index: u64,
        pub lowlink: u64,
        pub on_stack: bool,
    }
\end{lstlisting}

\begin{itemize}
    \item \textbf{vertex}: This attribute holds the original vertex value from the graph.
    It uniquely identifies the vertex this `SCCNode' instance represents.
    \item \textbf{index}: This attribute holds the discovery index of the vertex.
    When a vertex is first visited, it is assigned a unique index, which essentially records the order of the vertex's discovery during the depth-first search (DFS).
    \item \textbf{lowlink}: This attribute is crucial for Tarjan's algorithm.
    The `lowlink' value of a vertex is the smallest index of any vertex that is reachable from the vertex, including the vertex itself.
    It helps in identifying the root of a strongly connected component.
    \item \textbf{on\_stack}: This attribute indicates whether the vertex is currently on the stack (`dfs\_stk') used by the algorithm.
    It is used to manage the backtracking process and to ensure that we only consider vertices that are part of the current path in the DFS\@.
\end{itemize}


\subsubsection{MiniGraph}

\begin{lstlisting}[language=Rust, style=rust, caption={Defining the Minigraph with Rust}, label={lst:rust-define-minigraph}, firstnumber=1]
    #[derive(Debug)]
    pub struct MiniGraph {
        dico: HashMap<u64, HashSet<u64>>,
    }
\end{lstlisting}

\subsubsection{DGraph}

\begin{lstlisting}[language=Rust, style=rust, caption={Defining the DGraph with Rust}, label={lst:rust-define-dgraph}, firstnumber=1]
    #[derive(Debug)]
    pub(crate) struct DGraph {
        nodes: HashSet<usize>,
        edges: Vec<HashSet<usize>>,
        modules: Vec<HashSet<usize>>,
    }
\end{lstlisting}

\begin{itemize}
    \item \textbf{nodes}: A set of nodes in the graph.
    \item \textbf{edges}: A vector where each element is a set of nodes representing edges for each node.
    \item \textbf{modules}: A vector of sets, where each set contains nodes representing a module within the graph.
\end{itemize}

\subsubsection{ComponentGraph}

\begin{lstlisting}[language=Rust, style=rust, caption={Defining the Component Graph with Rust}, label={lst:rust-define-component-graph}, firstnumber=1]
    #[derive(Debug)]
    struct ComponentGraph {
        graph: DGraph,
        components: Vec<HashSet<usize>>,
        modules: Vec<HashSet<usize>>,
    }
\end{lstlisting}

\begin{itemize}
    \item \textbf{graph}: The underlying directed graph.
    \item \textbf{components}: A vector of sets, where each set contains nodes representing a strongly connected component.
    \item \textbf{modules}: A vector of sets, where each set contains nodes representing a module within the graph.
\end{itemize}

\subsubsection{TreeNode}

\begin{lstlisting}[language=Rust, style=rust, caption={Defining the TreeNode with Rust}, label={lst:rust-define-treenode}, firstnumber=1]
    #[derive(Debug)]
    struct TreeNode {
        node_type: String,
        node_colors: Vec<usize>,
        children: Vec<TreeNode>,
        node_id: Option<usize>,
        prime_graph: Option<TwoStructure>,
    }
\end{lstlisting}

\begin{itemize}
    \item \textbf{node\_type}: A string indicating the type of the node.
    \item \textbf{node\_colors}: A vector of colors associated with the node.
    \item \textbf{children}: A vector of `TreeNode` instances representing the children of this node.
    \item \textbf{node\_id}: An optional node identifier.
    \item \textbf{prime\_graph}: An optional `TwoStructure` representing a prime graph associated with this node.
\end{itemize}

\subsection{Algorithms, Methods and functions}\label{subsec:algoithms-methods-and-functions}

Each of these data structures defined above implements functionality.
Here, we'll look at the ones that are most important during the modular decomposition.

\subsubsection{Maximal modules}

The `maximal\_modules' function aims to find all maximal modules (subsets of nodes) in a graph that include a specific node `v'.

\begin{lstlisting}[language=Rust, style=rust, caption={Defining the maximal modules with Rust}, label={lst:rust-define-maximal-modules}, firstnumber=1]
    /// constructs M(g, v) according to the algorithm 3.1 of the article [1]
    ///
    /// # Arguments
    ///
    /// * 'g' - a two-structure
    /// * 'v' - a node of g
    ///
    /// # Returns
    ///
    /// a list of sets of nodes (modules)
    pub fn maximal_modules(g: &mut TwoStructure, v: u64) -> Vec<HashSet<u64>> {
        let mut init_module = g.nodes.clone();
        init_module.remove(&v);
        let mut partition = Vec::new();
        partition.push(init_module);
        let mut outsiders = Vec::new();
        let mut outsiders_h = HashSet::new();
        outsiders_h.insert(v);
        outsiders.push(outsiders_h);
        loop {
            if partition.is_empty() {
                panic!("Empty partition (please report)")
            }
            let mut module = HashSet::new();
            let mut index = 0;
            for i in 0..partition.len() {
                if outsiders.get(i).unwrap().len() as u64 > 0 {
                    // pick a module with outsiders
                    module = partition.get(i).unwrap().clone();
                    index = i;
                    break;
                }
            }
            if module.is_empty() {
                // if there is no module left having outsiders
                return partition;
            }

            // partition.remove(module)
            for i in 0..partition.len() {
                if *partition.get(i).unwrap() == module {
                    partition.remove(i);
                    break;
                }
            }

            let outsiders_i = outsiders.remove(index);
            // outsiders.remove(outsiders_i)


            // pick an arbitrary element
            let w = pickup(outsiders_i.clone());

            // list of sets of nodes
            let mut graph_alter = g.clone();
            let module_partition = module.clone();
            let r_modules = partition_module(&mut graph_alter, w, module_partition);
            for r_module in r_modules {
                let r_module_diff = r_module.clone();
                partition.push(r_module);
                // a = module - r_module
                let ha_alter = module.clone();
                let ha = ha_alter
                    .difference(&r_module_diff)
                    .collect::<HashSet<_>>();
                let mut a = HashSet::new();
                for &eha in ha {
                    a.insert(eha);
                }
                let mut b: HashSet<u64> = HashSet::new();
                if !outsiders_i.is_empty() {
                    for el in outsiders_i.iter() {
                        b.insert(*el);
                    }
                }
                b.remove(&w);
                outsiders.push(b);
            }
        }

        // this never happens
        // partition
    }
\end{lstlisting}

In arguments, we have:
\begin{itemize}
    \item `g': A reference to a `TwoStructure' representing the graph.
    \item `v': A node in the graph used as a reference for finding maximal modules.
\end{itemize}

and return a `Vec\textless HashSet\textless usize\textgreater\textgreater' where each `HashSet\textless usize\textgreater' represents a maximal module containing the node `v'.

\subsubsection{Partition module}

The `partition\_module' function aims to divide a given module (subset of nodes) into smaller submodules based on certain criteria related to the relationships between nodes in a graph.

\begin{lstlisting}[language=Rust, style=rust, caption={Defining the partition module with Rust}, label={lst:rust-define-partition module}, firstnumber=1]
    /// Partitions module in strong maximal modules
    ///
    /// # Arguments
    ///
    /// * 'g' - a two-structure
    /// * 'w' - the first node used for partitioning module
    /// * 'module' - a set of nodes of g to partition
    ///
    /// # Returns
    ///
    /// a list of sets of nodes
    pub fn partition_module(g: &mut TwoStructure, w: u64, module: HashSet<u64>) -> Vec<HashSet<u64>> {
        let colors_len = g.colors.len();

        // modules = [ [set() for c in g.colors] for c_ in g.colors ]
        let mut modules: Vec<Vec<HashSet<u64>>> = Vec::new();

        // is_empty = [ [True for c in g.colors] for c_ in g.colors ]
        let mut is_empty: Vec<Vec<bool>> = Vec::new();
        for _ in 0..colors_len {
            let mut vec_intern_modules_aux: Vec<HashSet<u64>> = Vec::new();
            let mut vec_intern_is_empty_aux: Vec<bool> = Vec::new();
            for _ in 0..colors_len {
                let set_inter_modules_aux: HashSet<u64> = HashSet::new();
                vec_intern_modules_aux.push(set_inter_modules_aux);
                vec_intern_is_empty_aux.push(false);
            }
            modules.push(vec_intern_modules_aux);
            is_empty.push(vec_intern_is_empty_aux);
        }

        for n in module {
            let c_w_n = g.color_of(w, n); // first color
            let c_n_w = g.color_of(n, w); // second color

            modules.get_mut(c_w_n as usize).unwrap().get_mut(c_n_w as usize).unwrap().insert(n);

            if *is_empty.get(c_w_n as usize)
                .unwrap()
                .get(c_n_w as usize)
                .unwrap()
            {
                is_empty.get_mut(c_w_n as usize)
                    .unwrap()
                    .push(false);
            }
        }

        let mut flattened: Vec<HashSet<u64>> = Vec::new();
        for m in modules {
            for s in m {
                if !s.is_empty() {
                    flattened.push(s);
                }
            }
        }
        flattened
    }
\end{lstlisting}

In Arguments, we have:
\begin{itemize}
    \item `g': A reference to a `TwoStructure' representing the graph.
    \item `w': A node in the graph used as a reference for partitioning.
    \item `module': A reference to a `HashSet<usize>' representing the module (subset of nodes) to be partitioned.
\end{itemize}

and return a `Vec\textless HashSet\textless usize\textgreater\textgreater' where each `HashSet\textless usize\textgreater' represents a submodule resulting from the partitioning process.


\subsubsection{Modular decomposition}

The `modular\_decomposition' function aims to decompose a graph into its modular components using a specified total order if provided.
It returns a hierarchical structure representing the modular decomposition.

\begin{lstlisting}[language=Rust, style=rust, caption={Defining the modular decomposition with Rust}, label={lst:rust-define-modular-decomposition}, firstnumber=1]
    fn modular_decomposition(g: &mut TwoStructure, total_order: Option<Vec<u64>>) -> TreeNode {
        if g.nodes.len() == 1 {
            let mut root = TreeNode::new();
            root.node_type = "LEAF".to_string();
            let iter = g.nodes.iter();
            let cloned_set: HashSet<u64> = iter.cloned().collect();
            root.node_id = Some(pickup(cloned_set));
            return root;
        }

        let mut total_order = total_order.unwrap_or_else(|| compute_total_order(g));
        total_order.retain(|&n| g.nodes.contains(&n));
        if total_order.is_empty() {
            total_order = compute_total_order(g);
        }

        let mut v = None;
        while let Some(node) = total_order.pop() {
            if g.nodes.contains(&node) {
                v = Some(node);
                break;
            }
        }
        let v = v.expect("No valid node found");

        let mods = maximal_modules(g, v);
        let mut qg = build_quotient(g, mods, v);
        let dg = build_distinction_graph(&mut qg, v);
        let mut cg = ComponentGraph::new(dg);

        let mut root = TreeNode::new();
        let mut u = &mut root;
        u.prime_graph = Some(qg.clone());

        while !cg.is_empty() {
            let mut w = TreeNode::new();
            w.node_type = "LEAF".to_string();
            w.node_id = Some(v);
            u.add_child(w);

            // get a mutable reference to the last child
            let u_ref = u;
            u = u_ref.children.last_mut().unwrap();

            let sinks = cg.remove_sinks();
            let mut fmodules: HashSet<u64> = HashSet::new();
            for sink in &sinks {
                for &mod_ in sink {
                    fmodules.insert(mod_);
                }
            }

            if sinks.len() == 1 && fmodules.len() > 1 {
                u.node_type = "PRIME".to_string();
                u.prime_graph = Some(qg.clone());
            } else {
                let iter = fmodules.iter();
                let p = pickup(iter.cloned().collect());
                let ntm = qg.node_to_module(p);
                match ntm {
                    Some(value) => {
                        let x = pickup((*value).clone());
                        if g.color_of(x, v) == g.color_of(v, x) {
                            u.node_type = "COMPLETE".to_string();
                            u.node_colors = vec![g.color_of(v, x)];
                        } else {
                            u.node_type = "LINEAR".to_string();
                            u.node_colors = vec![g.color_of(x, v), g.color_of(v, x)];
                        }
                    }
                    _ => {}
                }
            }

            for &fmod in &fmodules {
                let ntm = qg.node_to_module(fmod);
                match ntm {
                    Some(value) => {
                        let mut fg = g.slice((*value).clone());
                        let dtree = modular_decomposition(&mut fg, Some(total_order.clone()));
                        if ((u.node_type == "COMPLETE" && dtree.node_type == "COMPLETE")
                            && (u.node_colors == dtree.node_colors))
                            || ((u.node_type == "LINEAR" && dtree.node_type == "LINEAR")
                            && (u.node_colors == dtree.node_colors)) {
                            u.children.extend(dtree.children);
                        } else {
                            u.children.push(dtree);
                        }
                    }
                    _ => {}
                }
            }
        }
        root
    }
\end{lstlisting}

In arguments, we have:
\begin{itemize}
    \item `g': A mutable reference to a `TwoStructure' representing the graph to be decomposed.
    \item `total\_order': An optional vector of `u64' values specifying a total order of nodes.
    If provided, this order influences the decomposition process.
\end{itemize}

and return a `TreeNode' representing the hierarchical structure of the modular decomposition.

\hspace{4cm}


The code of all methods and functions defined in this stay in my GitHub repository~\cite{rustcode};


\section{Implementation in C++}\label{sec:implementation-in-c++}

C++ is a widely-used programming language known for its performance and control over system resources.
Here, he mainly helped me by giving me an approach to converting code from Python to Rust.
Firstly, it allowed me to properly type each variable in the Python code, then to differentiate between passing a variable by value or by reference when calling functions, and finally, it allowed me to clearly define the lifetime of the different variables declared in each function.

\subsection{Graph Representation in C++}\label{subsec:graph-representation-in-c++}

For the definition of the various graphs, we mainly use the same structures as those defined in the Rust code:

\subsubsection{2-structure}

\begin{lstlisting}[language=C++, style=cpp, caption={Defining the 2-Structure with C++}, label={lst:cpp-define-2-structure}, firstnumber=1]
    class TwoStructure
    {
    public:
        /// the graph of a two structure
        set<unsigned int> nodes;
        /// list of sets, colors[i] contains the set of edges of color i
        vector< set< tuple<unsigned int, unsigned int> > > colors;
        /// only for quotient graph, contains the nodes of the graph if they are modules they
        vector< set<unsigned int> > modules;

        TwoStructure() {
            set< tuple<unsigned int, unsigned int> > s;
            colors.push_back(s);
        }
    };
\end{lstlisting}

\subsubsection{SCCNode}

\begin{lstlisting}[language=C++, style=cpp, caption={Defining the SCCNode with C++}, label={lst:cpp-define-sccnode}, firstnumber=1]
    class SCCNode
    {
    public:
        unsigned int vertex;
        unsigned int index;
        unsigned int lowlink;
        bool on_stack;

        SCCNode(unsigned int v, unsigned int i) {
            vertex = v;
            index = i;
            lowlink = i;
            on_stack = false;
        }
    };
\end{lstlisting}

\subsubsection{SCC}

\begin{lstlisting}[language=C++, style=cpp, caption={Defining the SCC with C++}, label={lst:cpp-define-scc}, firstnumber=1]
    template <class T>
    class SCC
    {
    public:
        T* graph;
        unsigned int index;
        vector<set<unsigned int>> components;
        unsigned int comp_index;
        vector<SCCNode*> dfs_stk;
        map<unsigned int, SCCNode*> visited;

        SCC(T* g) {
            graph = g;
            index = 0;
            comp_index = 0;
        }
    };
\end{lstlisting}

\subsubsection{MiniGraph}

\begin{lstlisting}[language=C++, style=cpp, caption={Defining the MiniGraph with C++}, label={lst:cpp-define-mini-graph}, firstnumber=1]
    class MiniGraph
    {
    public:
        map<unsigned int, set<unsigned int>> dico;

        MiniGraph(map<unsigned int, set<unsigned int>> d) {
            dico = d;
        }
    };
\end{lstlisting}

\subsubsection{DGraph}

\begin{lstlisting}[language=C++, style=cpp, caption={Defining the DGraph with C++}, label={lst:cpp-define-d-graph}, firstnumber=1]
    class MiniGraph
    {
    public:
        map<unsigned int, set<unsigned int>> dico;

        MiniGraph(map<unsigned int, set<unsigned int>> d) {
            dico = d;
        }
    };
\end{lstlisting}

\subsubsection{ComponentGraph}

\begin{lstlisting}[language=C++, style=cpp, caption={Defining the ComponentGraph with C++}, label={lst:cpp-define-component-graph}, firstnumber=1]
    class ComponentGraph {
    public:
        DGraph graph;
        vector<set<unsigned int>> components;
        vector<set<unsigned int>> modules;

        ComponentGraph(DGraph g) {
           graph = g;
           components = strongly_connected_components(graph);
           modules = graph.modules;
        }
    };
\end{lstlisting}

\subsubsection{TreeNode}

\begin{lstlisting}[language=C++, style=cpp, caption={Defining the TreeNode with C++}, label={lst:cpp-define-tree-node}, firstnumber=1]
    class TreeNode {
    public:
        string node_type;
        unsigned int node_colors;
        vector<TreeNode> children;
        unsigned int node_id;
        TwoStructure prime_graph;
    };
\end{lstlisting}

\subsection{Functions in C++}\label{subsec:functions-in-c++}

\subsubsection{Maximal modules}

\begin{lstlisting}[language=C++, style=cpp, caption={Defining Maximale Module with C++}, label={lst:cpp-define-maximal-module}, firstnumber=1]
    vector<set<unsigned int>> maximal_modules(TwoStructure *g, unsigned int v) {
        set<unsigned int> init_module(g->nodes);
        init_module.erase(v);
        vector<set<unsigned int>> partition({init_module});
        vector<set<unsigned int>> outsiders({set<unsigned int>({v})});
        set<unsigned int> module;
        int cnt = 1;
        while (true) {
           if (partition.empty()) {
              throw "Empty partition (please report)";
           }
           // module = None
           module.clear();
           // bool none_module = true;
           unsigned int index = 0;
           for (unsigned int i = 0; i < partition.size(); i++) {
              if (!outsiders.at(i).empty()) {
                 // pick a module with outsiders
                 module = partition.at(i); // that's a set of nodes
                 // none_module = false;
                 index = i;
                 break;
              }
           }

           if (module.empty()) {
              // if there is no module left having outsiders
              return partition;
           }
           // partition.remove(module)
           auto pos_module =
              find(partition.begin(), partition.end(), module);
           if (pos_module != partition.end()) {
              partition.erase(pos_module);
           }
           set<unsigned int> outsiders_i = outsiders.at(index);
           // outsiders.remove(outsiders_i)
           auto pos_outsiders_i =
              find(outsiders.begin(), outsiders.end(), outsiders_i);
           if (pos_outsiders_i != outsiders.end()) {
              outsiders.erase(pos_outsiders_i);
           }
           unsigned int w = pickup(outsiders_i); // pick an arbitrary element

           vector<set<unsigned int>> rmodules =
              partition_module(g, w, module); // list of sets of nodes
           for (auto & rmodule : rmodules) {
              partition.push_back(rmodule);
              // a = module - rmodule
              set<unsigned int> a(module);
              for (unsigned int rm : rmodule) {
                 a.erase(rm);
              }
              /*
              if outsiders_i != None:
                 b = a.union(outsiders_i)
              else:
                 b = a
              */
              set<unsigned int> b(a);
              if (!outsiders_i.empty()) {
                 for (unsigned int o : outsiders_i) {
                    b.insert(o);
                 }
              }
              b.erase(w);
              outsiders.push_back(b);
           }
        }
        return partition; // this never happens
    }

\end{lstlisting}

\subsubsection{Partition module}

\begin{lstlisting}[language=C++, style=cpp, caption={Defining the Partition Module with C++}, label={lst:cpp-define-partition-module}, firstnumber=1]
    vector<set<unsigned int>> partition_module(TwoStructure *g, unsigned int w, const set<unsigned int>& module) {
        unsigned int colors_len = g->colors.size();
        unsigned int i, j;

        // modules = [ [set() for c in g.colors] for c_ in g.colors ]
        vector<vector<set<unsigned int>>> modules(
           colors_len, vector<set<unsigned int>>(colors_len, set<unsigned int>()));

        // is_empty = [ [True for c in g.colors] for c_ in g.colors ]
        vector<vector<bool>> is_empty(colors_len, vector<bool>(colors_len, true));

        for (unsigned int n : module) {
           unsigned int c_w_n = g->color_of(w, n); // first color
           unsigned int c_n_w = g->color_of(n, w); // second color

           modules.at(c_w_n).at(c_n_w).insert(n);

           if (is_empty.at(c_w_n).at(c_n_w)) {
              is_empty.at(c_w_n).at(c_n_w) = false;
           }
        }

        // remove empty set
        vector<set<unsigned int>> flattened;
        for (auto & module : modules) {
           for (auto & s : module) {
              if (!s.empty()) {
                 flattened.push_back(s);
              }
           }
        }
        return flattened;
    }
\end{lstlisting}

\subsubsection{Modular decomposition}

\begin{lstlisting}[language=C++, style=cpp, caption={Defining the Modular Decomposition with C++}, label={lst:cpp-define-modular-decomposition}, firstnumber=1]
    TreeNode modular_decomposition(TwoStructure g) {
        if (g.nodes.size() <= 1) {
           TreeNode root;
           root.node_type = "LEAF";
           root.node_id = pickup(g.nodes);
           return root;
        }


        // select an arbitrary node of g
        unsigned int v = pickup(g.nodes);

        // Compute the maximal modules of g wrt. v
        vector<set<unsigned int>> mods = maximal_modules(&g, v);

        // Compute the quotient structure
        TwoStructure qg = build_quotient(g, &mods, v);


        // Compute the 'distinguish' (?) graph wrt. the image of v
        DGraph dg = build_distinction_graph(qg, &v);

        // Compute the component graph
        ComponentGraph cg(dg);


        TreeNode root;
        TreeNode *u = &root;
        // u->quotient = qg;
        while (!cg.is_empty()) {
           TreeNode w;
           u->add_child(w);
           w.node_type = "LEAF";
           w.node_id = v; // line missing //

           vector<set<unsigned int>> sinks = cg.remove_sinks();
           set<unsigned int> fmodules;

           for (auto & sink : sinks) {
              for (unsigned int mod : sink) {
                 fmodules.insert(mod);
              }
           }

           if (sinks.size() == 1 && fmodules.size() > 1) {  // Set the type of node u
              u->node_type = "PRIME"; // one complete connected component
              u->prime_graph = qg;
           } else {
                cout << "g.nodes().size() = " << fmodules.size() << endl;
              unsigned int x = pickup(qg.node_to_module(pickup(fmodules)));
              if (g.color_of(x, v) == g.color_of(v, x)) {
                 u->node_type = "COMPLETE"; // sink is one
                 u->node_colors = g.color_of(v, x);
              } else {
                 u->node_type = "LINEAR"; // sinks maybe one or more
                 u->node_colors = g.color_of(x, v);
                 // dans le code python, on a u.node_colors = {g.color_of(c, v),
                 // g.color_of(v, x)} u->node_colors est un entier représentant
                 // une couleur ou un ensemble ???
              }
           }

           // Recursive calls
           // fmodule == fmod
           for (unsigned int fmodule : fmodules) {
              TwoStructure fg = g.slice(qg.node_to_module(fmodule));

              TreeNode dtree = modular_decomposition(fg);

              if (((u->node_type == "COMPLETE" && dtree.node_type == "COMPLETE") && (u->node_colors == dtree.node_colors))
                        || (u->node_type == "LINEAR" && dtree.node_type == "LINEAR") && (u->node_colors == dtree.node_colors)) {
                 for (auto & child : dtree.children) {
                    u->children.push_back(child);
                 }
              } else {
                 u->children.push_back(dtree);
              }
           }
           u = &w;
        }

        return root;
    }
\end{lstlisting}




% Benchmarking
    %! Author = adrien koumgang tegantchouang
%! Date = 09/07/24


\chapter{Benchmarking}\label{ch:benchmarking}

The performance tests will be carried out on three implementations: SageMath's, Pistiloglou's and my own.
To do this, I have chosen an approach that consists of calculating the time needed to perform the modular decomposition of a graph over two groups of graphs.
The first group is made up of so-called small graphs, where the number of nodes is of the order of ten.
These are mainly the graphs discussed in the theory in Chapter~\ref{ch:modular-decomposition}.
The second group is made up of large, undirected, randomly generated graphs.
For this group, the number of nodes is on the order of multiples of hundreds.

\section{Development and test environment}\label{sec:development-and-test-environment}

To write the code and run the tests, I'm using a MacBook Pro of Apple with an M2 processor, 8GB of RAM and 512GB of ROM\cite{macbookprom2}.

\subsection{Development environment: RustRover}\label{subsec:development-environment-rustrover}

RustRover\cite{rustrover} by JetBrains\cite{jetbrains} is the chosen IDE for this project due to its powerful features tailored for Rust development.
It provides advanced code navigation, refactoring tools, debugging capabilities, and seamless integration with Rust's toolchain.
This environment allows for efficient handling of both small and large Rust projects, making it ideal for modular decomposition.

Key features of RustRover that support this project:
\begin{itemize}
    \item \textbf{Intelligent Code Assistance:} Code completion, suggestions, and real-time analysis.
    \item \textbf{Cargo Integration:} RustRover's deep integration with Cargo enables seamless management of dependencies, builds, and test runs.
    \item \textbf{Debugging:} Powerful debugger that supports breakpoints, variable inspection, and expression evaluation.
    \item \textbf{Version Control Integration:} Built-in Git and GitHub integration to handle version control without leaving the IDE\@.
    \item \textbf{Test Runner:} Simplifies the process of writing and running unit tests using Rust’s built-in test framework.
\end{itemize}

\subsection{Test environment and benchmarking}\label{subsec:test-environment-and-benchmarking}

The graphs on which the tests were carried out were generated at random using the following Python code and saved in .dot files which are then read by each implementation.

\begin{lstlisting}[language=Python, style=python, caption={Graph generation code}, label={lst:graph-generation-code}, firstnumber=1]
    import random

    from twostructure import TwoStructure


    def generate_graph_twostructure(n, p=0.1):
        import random
        ts = TwoStructure()
        for i in range(n):
            for j in range(i + 1, n):
                if random.random() < p:  # Add edges with probability p
                    ts.uedge(i, j)

        return ts
\end{lstlisting}


\subsubsection{Python tool for test benchmarking: pytest-benchmark}

Pytest-benchmark\cite{pytestbenchmark} is a plugin for the pytest framework that allows developers to benchmark the performance of their code.
It helps measure execution time and detect performance regressions across test runs.
Key features of pytest-benchmark include:
\begin{enumerate}
    \item \textbf{Performance Testing:} Easily measure and compare the speed of code sections or functions, providing detailed performance metrics like min, max, mean, and standard deviation of execution time.
    \item \textbf{Comparison:} Track and compare benchmark results over time to detect any performance regressions or improvements.
    \item \textbf{Customizable:} Users can control warm-up times, repetitions, and iterations to fine-tune the precision and accuracy of the benchmarks.
    \item \textbf{Reporting:} pytest-benchmark generates detailed, human-readable reports with performance data, which can be saved to files (JSON, CSV) for further analysis or CI integration.
    \item \textbf{Command-Line Integration:} Benchmarking results can be viewed and compared directly from the command line.
    \item \textbf{Integration with pytest:} pytest-benchmark fits seamlessly into existing pytest test suites, so you can combine functional testing with performance benchmarks.
\end{enumerate}


\subsubsection{Rust tool for test benchmarking: Criterion}

Criterion\cite{criterion} is a powerful and flexible benchmarking framework for Rust, which provides statistically rigorous and reliable performance metrics.
Criterion.rs benchmarks collect and store statistical information from run to run and can automatically detect performance regressions as well as measuring optimizations.
It helps to measure the performance of various code sections, especially critical algorithms like those used in this project for modular decomposition of graphs.

Criterion was chosen for this project due to its advanced features, including:
\begin{itemize}
    \item \textbf{Statistical Significance:} It ensures that the benchmarking results are reliable by conducting statistical analysis of the timings.
    \item \textbf{Visual Reports:} Criterion generates detailed reports in both text and HTML formats, allowing for easy interpretation of results.
    \item \textbf{Comparative Benchmarking:} It allows comparing current benchmarks with previous runs to track performance improvements or regressions.
    \item \textbf{Ease of Use:} Criterion integrates well with Cargo, making it simple to run benchmarks alongside tests.
\end{itemize}

The benchmarks focus on measuring the performance of key functions in the modular decomposition algorithm.
The Criterion framework is used to evaluate execution time, memory usage, and other metrics across different input sizes and graph configurations.

To set up Criterion for this project, the following steps were followed:
\begin{enumerate}
    \item Add Criterion as a development dependency in `Cargo.toml'
    \item  Create a `benches' directory in the project root, and add a new Rust file, typically named `benchmark.rs'.
    This file contains the benchmarking code.
    \item Ensure that benchmarks are placed in the `benches' directory, as Criterion uses the convention of placing benchmarks outside the main `src' directory to avoid them being compiled into the release builds.
\end{enumerate}

\begin{myex}[Example Benchmark Code for Rust]

\end{myex}

\begin{lstlisting}[language=Rust, style=rust, caption={Example of benchmark code for modular decomposition}, label={lst:rust-example-of-benchmark-code}, firstnumber=1]
    use std::hint::black_box;
    use criterion::{criterion_group, criterion_main, Criterion};

    use moddecomp::two_structure::{graph_ex1, TwoStructure};
    use moddecomp::moddecomp::moddecomp::modular_decomposition;


    fn criterion_benchmark(c: &mut Criterion) {
        c.bench_function("modular decomposition graph ex1", |b| b.iter(|| modular_decomposition(black_box(&mut graph_ex1()), black_box(None))));
    }

    criterion_group!(benches, criterion_benchmark);
    criterion_main!(benches);
\end{lstlisting}

\begin{itemize}
    \item \textbf{black\_box:} Ensures that the Rust compiler does not optimize away the benchmarked code, providing more accurate results.
    \item \textbf{criterion\_group and criterion\_main:} These macros register the benchmarks and allow Criterion to execute them.
\end{itemize}

Once Criterion is configured, benchmarks can be run using Cargo: `cargo bench'.
This command will execute the benchmarks, and Criterion will output the results in the terminal.
It will also generate more detailed reports in the `target/criterion' directory, including both text summaries and HTML reports for visual analysis.

The benchmarks are designed to evaluate the following:
\begin{itemize}
    \item \textbf{Execution Time:} How long it takes for the modular decomposition algorithm to run on different types and sizes of graphs.
    \item \textbf{Scalability:} How the algorithm’s performance changes as the size and complexity of the graph input increase.
    \item \textbf{Comparative Performance:} How different versions of the algorithm or alternative approaches perform relative to each other.
\end{itemize}

Benchmarks are run with multiple iterations and warm-up phases to ensure accuracy.
Criterion’s statistical analysis methods (including bootstrapping) are used to minimize variance and provide a robust estimate of the true performance characteristics.

\begin{myex}[Example output]
    For the tests, we will be looking mainly at two output results:
    \begin{enumerate}
        \item \textbf{Terminal Output:}
                \begin{figure}[!h]
                    \centering
                    \includegraphics[width=0.80\textwidth]{images/benchmark/benchmark-terminal-output}
                    \caption{Example of Terminal ouput for modular decomposition of graph ex1}
                    \label{fig:example-of-terminal-output}
                \end{figure}
        \item \textbf{HTML Reports:}
                \begin{figure}[!h]
                    \centering
                    \includegraphics[width=1\textwidth]{images/benchmark/benchmark-html-output}
                    \caption{Example of HTML ouput for modular decomposition of graph ex1}
                    \label{fig:example-of-html-output}
                \end{figure}
    \end{enumerate}
\end{myex}


\subsubsection{Python code for test benchmarking SageMath implementation}

Unfortunately, I haven't found a tool that lets me run benchmark tests directly in SageMath.
To replicate the style of results we get from Criterion in Rust and pytest-benchmark in Pythib, I use the following approach in SageMath:
\begin{itemize}
    \item \textbf{Multiple Timings and Averages:} Run the code multiple times to get an average execution time.
    \item \textbf{Statistical Reporting:} Report min, max, and mean execution times, along with standard deviation, similar to Criterion’s output.
\end{itemize}

\begin{myex}[Example Benchmark Code SageMath]

\end{myex}

\begin{lstlisting}[language=Python, style=python, caption={Example of benchmark code for modular decomposition}, label={lst:sagemath-example-of-benchmark-code}, firstnumber=1]
    import time
    import numpy as np

    def benchmark_modular_decomposition(graph, iterations=100):
        times = []

        for _ in range(iterations):
            # Start timer
            start_time = time.time()

            # Perform the computation
            graph.modular_decomposition()

            # Stop timer
            end_time = time.time()

            # Record execution time in microseconds (µs)
            execution_time = (end_time - start_time)
            times.append(execution_time)

        # Calculate benchmark statistics using NumPy
        min_time = np.min(times)
        max_time = np.max(times)
        mean_time = np.mean(times)
        stdev_time = np.std(times)

        # Print results in seconds (s)
        print(f"Benchmark Results (over {iterations} iterations) in s:")
        print(f"Min Time: {min_time:.2f} s")
        print(f"Max Time: {max_time:.2f} s")
        print(f"Mean Time: {mean_time:.2f} s")
        print(f"Standard Deviation: {stdev_time:.2f} s")
\end{lstlisting}


\section{Result for small graphs}\label{sec:result-for-small-graphs}

Let's start with the first test on the graph in Figure\ref{fig:example-undirected-graph} and in Figure\ref{fig:example-directed-graph}.
Here it's called a small graph because of its size, containing just 11 nodes for the first graph and 8 for the second.

\newpage

\subsection{Benchmark for simple undirected graph\ref{fig:example-undirected-graph}}\label{subsec:benchmark-for-simple-undirected-graph}

\subsubsection*{SageMath implementation}
\begin{figure}[!h]
    \centering
    \includegraphics[width=0.40\textwidth]{images/benchmark/graph_wikipedia/benchmark_graph_wikipedia_sagemath}
    \caption{Result benchmark SageMath implementation}
    \label{fig:benchmark-graph-wikipedia-sagemath}
\end{figure}

\subsubsection*{Eleni Pistiloglou's implementation}
\begin{figure}[!h]
    \centering
    \includegraphics[width=0.60\textwidth]{images/benchmark/graph_wikipedia/benchmark_graph_wikipedia_python}
    \caption{Result benchmark Eleni Pistiloglou's implementation}
    \label{fig:benchmark-graph-wikipedia-python}
\end{figure}

\subsubsection*{Rust implementation}
\begin{figure}[!h]
    \centering
    \includegraphics[width=0.40\textwidth]{images/benchmark/graph_wikipedia/benchmark_graph_wikipedia_rust}
    \caption{Result benchmark Rust implementation}
    \label{fig:benchmark-graph-wikipedia-rust}
\end{figure}

\subsubsection*{Benchmark Comparison}
\begin{figure}[!h]
    \centering
    \includegraphics[width=0.40\textwidth]{images/benchmark/graph_wikipedia/benchmark_comparison_graph_wikipedia}
    \caption{Result benchmark Rust implementation}
    \label{fig:benchmark-comparison-graph-wikipedia}
\end{figure}

\newpage

\subsection{Benchmark for simple directed graph\ref{fig:example-directed-graph}}\label{subsec:benchmark-for-simple-directed-graph}

\subsubsection*{SageMath implementation}
\begin{figure}[!h]
    \centering
    \includegraphics[width=0.40\textwidth]{images/benchmark/digraph/benchmark_digraph_sagemath}
    \caption{Result benchmark SageMath implementation}
    \label{fig:benchmark-digraph-sagemath}
\end{figure}

\subsubsection*{Eleni Pistiloglou's implementation}
\begin{figure}[!h]
    \centering
    \includegraphics[width=0.60\textwidth]{images/benchmark/digraph/benchmark_digraph_python}
    \caption{Result benchmark Python implementation}
    \label{fig:benchmark-digraph-python}
\end{figure}

\subsubsection*{Rust implementation}
\begin{figure}[!h]
    \centering
    \includegraphics[width=0.40\textwidth]{images/benchmark/digraph/benchmark_digraph_rust}
    \caption{Result benchmark Rust implementation}
    \label{fig:benchmark-digraph-rust}
\end{figure}

\subsubsection*{Benchmark Comparison}
\begin{figure}[!h]
    \centering
    \includegraphics[width=0.40\textwidth]{images/benchmark/digraph/benchmark_comparison_digraph}
    \caption{Result benchmark Rust implementation}
    \label{fig:benchmark-comparison-digraph}
\end{figure}


\newpage


\section{Result for large graphs}\label{sec:result-for-large-graphs}

\subsection{100 nodes and 1058 edges}\label{subsec:result-for-graphs-100-1058}

\subsubsection*{SageMath implementation}
\begin{figure}[!h]
    \centering
    \includegraphics[width=0.40\textwidth]{images/benchmark/large_graph/benchmark_large_graph_sagemath}
    \caption{Result benchmark SageMath implementation}
    \label{fig:benchmark-large-graph-sagemath}
\end{figure}

\subsubsection*{Eleni Pistiloglou's implementation}
\begin{figure}[!h]
    \centering
    \includegraphics[width=0.50\textwidth]{images/benchmark/large_graph/benchmark_large_graph_python}
    \caption{Result benchmark Eleni Pistiloglou's implementation}
    \label{fig:benchmark-large-graph-python}
\end{figure}

\subsubsection*{Rust implementation}
\begin{figure}[!h]
    \centering
    \includegraphics[width=0.40\textwidth]{images/benchmark/large_graph/benchmark_large_graph_rust}
    \caption{Result benchmark Rust implementation}
    \label{fig:benchmark-large-graph-rust}
\end{figure}

\subsubsection*{Benchmark Comparison}
\begin{figure}[!h]
    \centering
    \includegraphics[width=0.35\textwidth]{images/benchmark/large_graph/benchmark_comparison_graph_100_1058}
    \caption{Result benchmark Rust implementation}
    \label{fig:benchmark-comparison-graph-100-1058}
\end{figure}


\newpage


\subsection{100 nodes and 970 edges}\label{subsec:result-for-graphs-100-970}

\subsubsection*{SageMath implementation}
\begin{figure}[!h]
    \centering
    \includegraphics[width=0.40\textwidth]{images/benchmark/graph_100_970/benchmark_graph_100_970_sagemath}
    \caption{Result benchmark SageMath implementation}
    \label{fig:benchmark-graph-100-970-sagemath}
\end{figure}

\subsubsection*{Eleni Pistiloglou's implementation}
\begin{figure}[!h]
    \centering
    \includegraphics[width=0.60\textwidth]{images/benchmark/graph_100_970/benchmark_graph_100_970_python}
    \caption{Result benchmark Eleni Pistiloglou's implementation}
    \label{fig:benchmark-graph-100-970-python}
\end{figure}

\subsubsection*{Rust implementation}
\begin{figure}[!h]
    \centering
    \includegraphics[width=0.40\textwidth]{images/benchmark/graph_100_970/benchmark_graph_100_970_rust}
    \caption{Result benchmark Rust implementation}
    \label{fig:benchmark-graph-100-970-rust}
\end{figure}

\subsubsection*{Benchmark Comparison}
\begin{figure}[!h]
    \centering
    \includegraphics[width=0.40\textwidth]{images/benchmark/graph_100_970/benchmark_comparison_graph_100_970}
    \caption{Result benchmark Rust implementation}
    \label{fig:benchmark-comparison-graph-100-970}
\end{figure}


\newpage


\subsection{200 nodes and 3896 edges}\label{subsec:result-for-graphs-200-3896}

\subsubsection*{SageMath implementation}
\begin{figure}[!h]
    \centering
    \includegraphics[width=0.40\textwidth]{images/benchmark/graph_200_3896/benchmark_graph_200_3896_sagemath}
    \caption{Result benchmark SageMath implementation}
    \label{fig:benchmark-graph-200-3896-sagemath}
\end{figure}

\subsubsection*{Eleni Pistiloglou's implementation}
\begin{figure}[!h]
    \centering
    \includegraphics[width=0.60\textwidth]{images/benchmark/graph_200_3896/benchmark_graph_200_3996_python}
    \caption{Result benchmark Eleni Pistiloglou's implementation}
    \label{fig:benchmark-graph-200-3896-python}
\end{figure}

\subsubsection*{Rust implementation}
\begin{figure}[!h]
    \centering
    \includegraphics[width=0.40\textwidth]{images/benchmark/graph_200_3896/benchmark_graph_200_3996_rust}
    \caption{Result benchmark Rust implementation}
    \label{fig:benchmark-graph-200-3896-rust}
\end{figure}

\subsubsection*{Benchmark Comparison}
\begin{figure}[!h]
    \centering
    \includegraphics[width=0.40\textwidth]{images/benchmark/graph_200_3896/benchmark_comparison_graph_200_3896}
    \caption{Result benchmark Rust implementation}
    \label{fig:benchmark-comparison-graph-200-3896}
\end{figure}

\newpage


\subsection{200 nodes and 3880 edges}\label{subsec:result-for-graphs-200-3880}

\subsubsection*{SageMath implementation}
\begin{figure}[!h]
    \centering
    \includegraphics[width=0.40\textwidth]{images/benchmark/graph_200_3880/benchmark_graph_200_3880_sagemath}
    \caption{Result benchmark SageMath implementation}
    \label{fig:benchmark-graph-200-3880-sagemath}
\end{figure}

\subsubsection*{Eleni Pistiloglou's implementation}
\begin{figure}[!h]
    \centering
    \includegraphics[width=0.60\textwidth]{images/benchmark/graph_200_3880/benchmark_graph_200_3880_python}
    \caption{Result benchmark Eleni Pistiloglou's implementation}
    \label{fig:benchmark-graph-200-3880-python}
\end{figure}

\subsubsection*{Rust implementation}
\begin{figure}[!h]
    \centering
    \includegraphics[width=0.40\textwidth]{images/benchmark/graph_200_3880/benchmark_graph_200_3880_rust}
    \caption{Result benchmark Rust implementation}
    \label{fig:benchmark-graph-200-3880-rust}
\end{figure}

\subsubsection*{Benchmark Comparison}
\begin{figure}[!h]
    \centering
    \includegraphics[width=0.40\textwidth]{images/benchmark/graph_200_3880/benchmark_comparison_graph_200_3880}
    \caption{Result benchmark Rust implementation}
    \label{fig:benchmark-comparison-graph-200-3880}
\end{figure}


\newpage

\subsection{500 nodes - 24876 edges}\label{subsec:result-for-graphs-500-24876}

\subsubsection*{SageMath implementation}
\begin{figure}[!h]
    \centering
    \includegraphics[width=0.40\textwidth]{images/benchmark/too_large_graph/benchmark_too_large_graph_sagemath}
    \caption{Result benchmark SageMath implementation}
    \label{fig:benchmark-graph-500-24876-sagemath}
\end{figure}

\subsubsection*{Eleni Pistiloglou's implementation}
\begin{figure}[!h]
    \centering
    \includegraphics[width=0.60\textwidth]{images/benchmark/too_large_graph/benchmark_too_large_graph_python}
    \caption{Result benchmark Eleni Pistiloglou's implementation}
    \label{fig:benchmark-graph-500-24876-python}
\end{figure}

\subsubsection*{Rust implementation}
\begin{figure}[!h]
    \centering
    \includegraphics[width=0.40\textwidth]{images/benchmark/too_large_graph/benchmark_graph_500_24876_rust}
    \caption{Result benchmark Rust implementation}
    \label{fig:benchmark-graph-500-24876-rust}
\end{figure}

\subsubsection*{Benchmark Comparison}
\begin{figure}[!h]
    \centering
    \includegraphics[width=0.40\textwidth]{images/benchmark/too_large_graph/benchmark_comparison_graph_500_24876}
    \caption{Result benchmark Rust implementation}
    \label{fig:benchmark-comparison-graph-500-24876}
\end{figure}


\newpage


\subsection{500 nodes - 24864 edges}\label{subsec:result-for-graphs-500-24864}

\subsubsection*{SageMath implementation}
\begin{figure}[!h]
    \centering
    \includegraphics[width=0.60\textwidth]{images/benchmark/graph_500_24864/benchmark_graph_500_24864_sagemath}
    \caption{Result benchmark SageMath implementation}
    \label{fig:benchmark-graph-500-24864-sagemath}
\end{figure}

\subsubsection*{Eleni Pistiloglou's implementation}
\begin{figure}[!h]
    \centering
    \includegraphics[width=0.60\textwidth]{images/benchmark/graph_500_24864/benchmark_graph_500_24864_python}
    \caption{Result benchmark Eleni Pistiloglou's implementation}
    \label{fig:benchmark-graph-500-24864-python}
\end{figure}

\subsubsection*{Rust implementation}
\begin{figure}[!h]
    \centering
    \includegraphics[width=0.40\textwidth]{images/benchmark/graph_500_24864/benchmark_graph_500_24864_rust}
    \caption{Result benchmark Rust implementation}
    \label{fig:benchmark-graph-500-24864-rust}
\end{figure}

\subsubsection*{Benchmark Comparison}
\begin{figure}[!h]
    \centering
    \includegraphics[width=0.40\textwidth]{images/benchmark/graph_500_24864/benchmark_comparison_graph_500_24864}
    \caption{Result benchmark Rust implementation}
    \label{fig:benchmark-comparison-graph-500-24864}
\end{figure}


\newpage


\subsection{800 nodes - 63378 edges}\label{subsec:result-for-graphs-800-63378}

\subsubsection*{SageMath implementation}
\begin{figure}[!h]
    \centering
    \includegraphics[width=0.40\textwidth]{images/benchmark/graph_800_63378/benchmark_graph_800_63378_sagemath}
    \caption{Result benchmark SageMath implementation}
    \label{fig:benchmark-graph-800-63378-sagemath}
\end{figure}

\subsubsection*{Eleni Pistiloglou's implementation}
\begin{figure}[!h]
    \centering
    \includegraphics[width=0.60\textwidth]{images/benchmark/graph_800_63378/benchmark_graph_800_63378_python}
    \caption{Result benchmark Eleni Pistiloglou's implementation}
    \label{fig:benchmark-graph-800-63378-python}
\end{figure}

\subsubsection*{Rust implementation}
\begin{figure}[!h]
    \centering
    \includegraphics[width=0.40\textwidth]{images/benchmark/graph_800_63378/benchmark_graph_800_63378_rust}
    \caption{Result benchmark Rust implementation}
    \label{fig:benchmark-graph-800-63378-rust}
\end{figure}

\subsubsection*{Benchmark Comparison}
\begin{figure}[!h]
    \centering
    \includegraphics[width=0.40\textwidth]{images/benchmark/graph_800_63378/benchmark_comparison_graph_800_63378}
    \caption{Result benchmark Rust implementation}
    \label{fig:benchmark-comparison-graph-800-63378}
\end{figure}

\newpage


\subsection{1000 nodes - 100206 edges}\label{subsec:result-for-graphs-1000-100206}

\subsubsection*{SageMath implementation}
\begin{figure}[!h]
    \centering
    \includegraphics[width=0.80\textwidth]{images/benchmark/graph_1000_100206/benchmark_graph_1000_100206_sagemath}
    \caption{Result benchmark SageMath implementation}
    \label{fig:benchmark-graph-1000-100206-sagemath}
\end{figure}


\subsubsection*{Rust implementation}
\begin{figure}[!h]
    \centering
    \includegraphics[width=0.80\textwidth]{images/benchmark/graph_1000_100206/benchmark_graph_1000_100206_rust}
    \caption{Result benchmark Rust implementation}
    \label{fig:benchmark-graph-1000-100206-rust}
\end{figure}

\subsubsection*{Benchmark Comparison}
\begin{figure}[!h]
    \centering
    \includegraphics[width=0.40\textwidth]{images/benchmark/graph_1000_100206/benchmark_comparison_graph_1000_100206}
    \caption{Result benchmark Rust implementation}
    \label{fig:benchmark-comparison-graph-1000-100206}
\end{figure}


\newpage

\section{Conclusion}\label{sec:conclusion}

The benchmark results show that implementations of modular decomposition algorithms in SageMath, Eleni Pistiloglou's version in Python and Rust exhibit significant performance differences as a function of graph size and structure.

\begin{itemize}
    \item Small graphs: For small graphs, all implementations are efficient with minor differences in execution time.
    The Python and SageMath implementations gave comparable results, although SageMath was slightly faster.
    \item Large graphs: As the size of the graph increased, particularly for graphs containing hundreds of nodes and thousands of edges, the performance differences became pronounced.
    Rust always outperformed Python, thanks to its memory management and concurrency capabilities, but was falling behind SageMath.
    Python was the slowest performer, particularly when manipulating large graphs with complex structures, while SageMath performed moderately well, but often worse than Rust in some cases.
\end{itemize}

Overall, Rust offered good scalability and efficiency, particularly for smaller graphs.
SageMath, although slower, was better for larger and more complex graphs, and the Eleni Pistiloglou's implementation in Python, although flexible and easy to understand, struggled to perform on a large scale.


% Conclusion
    %! Author = adrien koumgang tegantchouang
%! Date = 09/07/24


\chapter{Conclusion}\label{ch:conclusion}

This thesis has delved into the intricate and powerful concept of modular decomposition in graph theory, focusing on both the theoretical framework and its practical implementations.
The modular decomposition of graphs is a fundamental technique that simplifies the structural analysis of complex networks by partitioning them into smaller, more manageable subgraphs called modules.
Through this process, computational problems in areas like bioinformatics, social network analysis, and communication networks become more tractable, allowing for more efficient algorithms to solve them.

A significant portion of the research was dedicated to the exploration and implementation of modular decomposition algorithms in different programming languages, namely Python, SageMath, Rust, and C++.
The decision to implement the algorithm in Rust stemmed from its growing reputation as a systems programming language that offers both performance and safety.
Rust’s strict ownership model, combined with its emphasis on memory safety without the need for a garbage collector, makes it ideal for performance-critical tasks such as graph decomposition on large datasets.

The Rust implementation, as demonstrated throughout this thesis, addressed some of the key performance bottlenecks identified in the earlier Python and SageMath versions.
One of the primary challenges in Python's implementation was the inefficiency in handling large graph structures due to its reliance on dynamic typing and high-level abstractions.
While Python is versatile and user-friendly, its inherent limitations in performance become evident when dealing with computationally intensive tasks.
SageMath, though designed for mathematical computations and capable of handling modular decomposition with greater efficiency than Python, still lagged behind in comparison to a lower-level implementation in Rust.

The transition from Python to Rust was not without its challenges.
Rust’s ownership model and its strict borrowing rules, while advantageous in terms of preventing memory errors, required careful handling during the implementation of recursive functions and complex data structures.
One of the key achievements of this thesis was the successful adaptation of modular decomposition algorithms to Rust’s stringent memory and concurrency requirements.
This process not only improved the overall performance of the algorithm but also reinforced the importance of safe memory management in large-scale computational tasks.

The C++ implementation that was developed in parallel turned out to present many more bugs than the Rust version, largely due to the complexities of manual memory management and the lack of built-in safety guarantees.
In C++, issues such as memory leaks, dangling pointers, and undefined behavior were more prevalent, requiring careful debugging and additional safeguards to ensure stability.
These problems arose because C++ does not enforce strict ownership and borrowing rules like Rust, making it easier to inadvertently introduce errors that compromise the integrity of the program.
In contrast, Rust’s ownership model and built-in safety features, such as the borrow checker, eliminated many of these potential issues at compile-time, resulting in cleaner, more reliable code with fewer runtime errors.
While C++ provides more flexibility and control over memory, this comes at the cost of increased complexity in maintaining bug-free code, especially in large-scale projects like graph decomposition.

Benchmarking results were a pivotal part of the research.
By testing the Rust implementation on various graph sizes, ranging from small graphs of 10 nodes to much larger graphs of 500 nodes, the thesis provided a comprehensive analysis of the algorithm’s scalability.
The results clearly showed that Rust’s performance significantly surpassed that of Python, especially as the size of the graphs increased.
For small graphs, the difference in execution time between Rust and Python was minimal, but as the graph size grew, Rust’s superior memory management and execution speed became more apparent.
This finding is crucial for applications in fields where large-scale graph processing is required, such as in bioinformatics or social network analysis, where datasets can include thousands or even millions of nodes and edges.

Additionally, the exploration of modular decomposition applied to 2-structures introduced an even more complex dimension to the problem.
The thesis extended traditional graph decomposition to 2-structures, which are characterized by colored arcs, making the decomposition process more sophisticated.
This additional complexity required further adaptation of the algorithm, which was successfully implemented in Rust, demonstrating the language's flexibility in handling complex data structures.

This thesis also lays the groundwork for further investigations.
Future work could focus on optimizing the algorithm even further, exploring parallelization techniques in Rust to handle even larger datasets or more complex graph structures.

Additionally, the modular decomposition algorithm could be extended to other types of graphs and applied to real-world datasets, such as those used in social network analysis or biological networks, to further validate its effectiveness and scalability.
An opportunity to resolve bug problems in the C++ version and test its performance against the Rust implementation.
The latter could well offer suggestions for improving the current version of Rust.
Ultimately, the modular decomposition of graphs, though rooted in theoretical graph theory, has broad implications across numerous domains.
As datasets continue to grow in size and complexity, the need for efficient and scalable algorithms will only increase, and the work presented in this thesis provides a valuable tool for researchers and engineers alike.

For me, this thesis work was very important because it gave me the opportunity to learn a very promising programming language, Rust.
Nowadays, I use Rust for other applications such as distributed systems, as well as for web programming.
It also enabled me to learn a lot about modular decomposition and the opportunities it offers in various areas of everyday life.
Like, for example, the work following mine that my teachers told me about, that of being able to optimise the execution of processes and threads in computers.
This could greatly increase the performance of the systems and applications we use every day.




% acknowledgement
    % 

\chapter*{Acknowledgments}\label{ch:acknowledgments}






    \bibliographystyle{plain}
    \bibliography{bibliography}



    \include{glossary}

\end{document}
