%! Author = adrien koumgang tegantchouang
%! Date = 09/07/24


\chapter{Implementation in Rust and C++}\label{ch:implementation-in-rust-and-cpp}

The implementation of the modular decomposition algorithm in Rust was motivated by the need to overcome performance limitations observed in previous implementation in Python.
The primary motivation for implementing the modular decomposition algorithm in Rust was to achieve better performance and scalability for large and complex graphs.
As described on the official Rust website, we have that Rust is blazingly fast and memory-efficient: with no runtime or garbage collector, it can power performance-critical services, run on embedded devices, and easily integrate with other languages.
Rust's rich type system and ownership model guarantee memory-safety and thread-safety -- enabling you to eliminate many classes of bugs at compile-time.
In brief, Rust is a programming language that adopts a very specific programming philosophy, mainly thanks to its ownership principle, which ensures memory safety without needing a garbage collector.
Ownership governs how memory is managed and how different parts of a program can access and modify data.

The key principles of ownership in Rust are~\cite{rust}:
\begin{itemize}
    \item \textbf{Ownership Rules}
    \begin{itemize}
        \item Each value in Rust has a variable that's called its owner.
        \item There can only be one owner at a time.
        \item When the owner goes out of scope, the value will be dropped (freed).
    \end{itemize}
    \item \textbf{Borrowing and References}
    \begin{itemize}
        \item A variable can borrow a value by creating a reference to it using `\&'.
        \item Borrowing can be either immutable (`\&T') or mutable (`\&mut T').
        \item Multiple immutable references are allowed, but only one mutable reference is allowed at a time.
    \end{itemize}
    \item \textbf{Move Semantics}
    \begin{itemize}
        \item When a value is assigned to another variable or passed to a function, the ownership of the value moves to the new variable or function parameter.
        \item After the move, the original variable is no longer accessible.
    \end{itemize}
    \item \textbf{Copy Trait}
    \begin{itemize}
        \item For types that implement the `Copy' trait (like integers and other simple types), a copy of the value is made rather than moving ownership.
    \end{itemize}
    \item \textbf{Lifetimes}
    \begin{itemize}
        \item Every ownership and every borrowing has a ``lifetime", which is the portion of the program along which it is active.
        \item Lifetimes are used by the Borrow Checker to check statically that the above rules are respected.
        \item Lifetimes ensure that references are valid as long as they are needed.
        \item They help the compiler reason about how long references should be valid and prevent dangling references.
    \end{itemize}
\end{itemize}

Coming as I do from programming languages such as C and Java, where you could have several pointers to a memory space in modification mode, where you could pass a pointer to a variable while still retaining control over it in write mode, Rust came to me as a real challenge.
The level of challenge was even greater because the code to be translated into Rust is written in Python~\cite{pythoncode}, which has the merit of being very flexible in terms of typing, variable access and memory areas~\cite{python}.
Many of the functions to be translated, especially the main one, were recursive, and some of them contained cycles that defined variables with a fairly complex life cycle.
As a result, I quickly ran into difficulties writing the code in Rust.
To overcome this, I turned to a programming language that has strongly influenced Rust: C++~\cite{cpp}.

This chapter provides an in-depth discussion of the methodologies, challenges and results associated with the implementation in Rust and the reasons why I decided to implement the algorithm also in C++.


\section{Implementation in Rust: Graph Representation, Algorithm Design and Optimization Techniques}\label{sec:implementation-in-rust:-graph-representation-algorithm-design-and-optimization-techniques}

This section delves into the detailed aspects of implementing the algorithm in Rust, covering the graph representation, algorithm design, and optimization techniques.

\subsection{Graph Representation}\label{subsec:graph-representation}

For the representation of the various graphs used during the execution of the modular decomposition algorithm, such as the 2-structures, I followed the implementation made by Eleni Pistiloglou in her second implementation in Python, which uses mainly sets instead of lists to optimise access to both nodes and arcs.

\subsubsection{2-structure}

We therefore have the following definition for the representation of a 2-structure:
\begin{lstlisting}[language=Rust, style=rust, caption={Defining the 2-structure}, label={lst:rust-define-twostructure}, firstnumber=1]
    #[derive(Debug)]
    #[derive(Clone)]
    pub struct TwoStructure {
        /// the graph of a two structure
        pub nodes: HashSet<u64>,
        /// list of sets, colors[i] contains the set of edges of color i
        pub colors: Vec<HashSet<(u64, u64)>>,
        /// only for quotient graph, contains the nodes of the graph if they are modules
        pub modules: Vec<HashSet<u64>>,
    }
\end{lstlisting}
Where we have:
\begin{itemize}
    \item \textbf{nodes}: the nodes of a two structure
    \item \textbf{colors}: list of sets, colors[i] contains the set of edges of color i
    \item \textbf{modules}: only for quotient graph, contains the nodes of the graph if they are modules
\end{itemize}

\subsubsection{SCC}

SCC, for Strongly Connected Components, is a structure that takes a graph as input and determines the `strongly connected components' of this graph using its two visit\ref{subsubsec:visit-for-scc-structure} and compute\ref{subsubsec:compute-for-scc-structure} functions.
For doing this, we use `Tarjan's strongly connected components algorithm'~\cite{sccalgowikipedia}.
Tarjan's strongly connected components algorithm is an algorithm in graph theory for finding the strongly connected components (SCCs) of a directed graph.
It runs in linear time, matching the time bound for alternative methods including Kosaraju's algorithm and the path-based strong component algorithm.
The algorithm is named for its inventor, Robert Tarjan.

\begin{lstlisting}[language=Rust, style=rust, caption={Defining the SCC}, label={lst:rust-define-scc}, firstnumber=1]
    pub struct SCC<'a> {
        pub graph: Box<dyn GraphTrait + 'a>,
        pub index: u64,
        pub components: Vec<HashSet<u64>>,
        pub comp_index: u64,
        pub dfs_stk: Vec<SCCNode>,
        pub visited: HashMap<u64, SCCNode>,
    }
\end{lstlisting}

\begin{itemize}
    \item \textbf{graph}: This attribute holds the graph on which Tarjan's algorithm will be executed.
    It provides the vertices and the adjacency list needed to traverse the graph.
    \item \textbf{index}: This attribute is used to assign a unique discovery index to each vertex when it is first visited during the depth-first search (`DFS').
    It is incremented with each new vertex visit.
    \item \textbf{components}: This attribute stores the strongly connected components (`SCCs') identified in the graph.
    Each SCC is represented as a set of vertices, and all such sets are collected in this list.
    \item \textbf{comp\_index}:
    \item \textbf{dfs\_stk}: This attribute acts as a stack to manage the depth-first search traversal.
    It helps in tracking the current path and managing backtracking, which is essential for updating the `lowlink' values and identifying SCCs.
    \item \textbf{visited}: This attribute keeps track of all the vertices that have been visited and their associated SCCNode instances.
    It helps in quickly checking whether a vertex has already been visited and accessing its SCCNode information.
\end{itemize}

where GraphTrait is:

\begin{lstlisting}[language=Rust, style=rust, caption={Defining GraphTrait}, label={lst:rust-define-graphtrait}, firstnumber=1]
    pub trait GraphTrait {
        fn vertices(&self) -> HashSet<u64>;
        fn successors(&mut self, v: u64) -> &HashSet<u64>;
    }
\end{lstlisting}

\subsubsection{SCCNode}

\begin{lstlisting}[language=Rust, style=rust, caption={Defining the SCCNode}, label={lst:rust-define-sccnode}, firstnumber=1]
    #[derive(Clone, Copy)]
    pub struct SCCNode {
        // original vertex ... must be hashable and unique (among vertices)
        pub vertex: u64,
        pub index: u64,
        pub lowlink: u64,
        pub on_stack: bool,
    }
\end{lstlisting}

\begin{itemize}
    \item \textbf{vertex}: This attribute holds the original vertex value from the graph.
    It uniquely identifies the vertex this `SCCNode' instance represents.
    \item \textbf{index}: This attribute holds the discovery index of the vertex.
    When a vertex is first visited, it is assigned a unique index, which essentially records the order of the vertex's discovery during the depth-first search (DFS).
    \item \textbf{lowlink}: This attribute is crucial for Tarjan's algorithm.
    The `lowlink' value of a vertex is the smallest index of any vertex that is reachable from the vertex, including the vertex itself.
    It helps in identifying the root of a strongly connected component.
    \item \textbf{on\_stack}: This attribute indicates whether the vertex is currently on the stack (`dfs\_stk') used by the algorithm.
    It is used to manage the backtracking process and to ensure that we only consider vertices that are part of the current path in the DFS\@.
\end{itemize}


\subsubsection{MiniGraph}

\begin{lstlisting}[language=Rust, style=rust, caption={Defining the Minigraph}, label={lst:rust-define-minigraph}, firstnumber=1]
    #[derive(Debug)]
    pub struct MiniGraph {
        dico: HashMap<u64, HashSet<u64>>,
    }
\end{lstlisting}

\subsubsection{DGraph}

\begin{lstlisting}[language=Rust, style=rust, caption={Defining the DGraph}, label={lst:rust-define-dgraph}, firstnumber=1]
    #[derive(Debug)]
    pub(crate) struct DGraph {
        nodes: HashSet<usize>,
        edges: Vec<HashSet<usize>>,
        modules: Vec<HashSet<usize>>,
    }
\end{lstlisting}

\begin{itemize}
    \item \textbf{nodes}: A set of nodes in the graph.
    \item \textbf{edges}: A vector where each element is a set of nodes representing edges for each node.
    \item \textbf{modules}: A vector of sets, where each set contains nodes representing a module within the graph.
\end{itemize}

\subsubsection{ComponentGraph}

\begin{lstlisting}[language=Rust, style=rust, caption={Defining the Component Graph}, label={lst:rust-define-component-graph}, firstnumber=1]
    #[derive(Debug)]
    struct ComponentGraph {
        graph: DGraph,
        components: Vec<HashSet<usize>>,
        modules: Vec<HashSet<usize>>,
    }
\end{lstlisting}

\begin{itemize}
    \item \textbf{graph}: The underlying directed graph.
    \item \textbf{components}: A vector of sets, where each set contains nodes representing a strongly connected component.
    \item \textbf{modules}: A vector of sets, where each set contains nodes representing a module within the graph.
\end{itemize}

\subsubsection{TreeNode}

\begin{lstlisting}[language=Rust, style=rust, caption={Defining the TreeNode}, label={lst:rust-define-treenode}, firstnumber=1]
    #[derive(Debug)]
    struct TreeNode {
        node_type: String,
        node_colors: Vec<usize>,
        children: Vec<TreeNode>,
        node_id: Option<usize>,
        prime_graph: Option<TwoStructure>,
    }
\end{lstlisting}

\begin{itemize}
    \item \textbf{node\_type}: A string indicating the type of the node.
    \item \textbf{node\_colors}: A vector of colors associated with the node.
    \item \textbf{children}: A vector of `TreeNode` instances representing the children of this node.
    \item \textbf{node\_id}: An optional node identifier.
    \item \textbf{prime\_graph}: An optional `TwoStructure` representing a prime graph associated with this node.
\end{itemize}

\subsection{Algorithms, Methods and functions}\label{subsec:algoithms-methods-and-functions}

Each of these data structures defined above implements functionality.
Here, we'll look at the ones that are most important during the modular decomposition.

\subsubsection{Compute for SCC structure}\label{subsubsec:compute-for-scc-structure}

The compute method is the entry point of the algorithm.
It iterates over all vertices of the graph and initiates the depth-first search (DFS) from any vertex that has not been visited yet.
The main goal of this method is to ensure that every node in the graph is processed and all SCCs are identified.

\begin{lstlisting}[language=Rust, style=rust, caption={Defining the TreeNode}, label={lst:rust-define-compute}, firstnumber=1]
    impl<'a> SCC<'a>
    {
        pub fn compute(&mut self) {
            for v in self.graph.vertices() {
                if self.visited.contains_key(&v) {
                    continue;
                }
                // v has not been visited
                let mut vnode = SCCNode::new(v, self.index);
                self.index += 1;
                vnode.on_stack = true;
                self.visited.insert(v, vnode);
                vnode.on_stack = false;
                self.visit(&RefCell::new(vnode));
            }
        }
    }
\end{lstlisting}

\subsubsection{Visit for SCC structure}\label{subsubsec:visit-for-scc-structure}

The visit method is where the depth-first search (DFS) happens.
It traverses the graph and updates the ``low-link” values that help determine the root of a strongly connected component.
The low-link value of a node is the smallest index reachable from that node, and it plays a critical role in identifying SCCs.

\begin{lstlisting}[language=Rust, style=rust, caption={Defining the Component Graph}, label={lst:rust-define-visit}, firstnumber=1]
    impl<'a> SCC<'a>
    {
        pub fn visit(&mut self, node: &RefCell<SCCNode>) -> SCCNode {
            let mut vnode = node.borrow_mut();
            
            vnode.on_stack = true;
            self.dfs_stk.push(*vnode);
    
            let s = self.graph.successors(vnode.vertex);
            match s {
                Some(h) => {
                    for w in h.clone() {
                        if !self.visited.contains_key(&w) {
                            // recurse
                            let mut wnode = SCCNode::new(w, self.index);
                            self.index += 1;
                            wnode.on_stack = true;
                            self.visited.insert(w, wnode);
                            wnode.on_stack = false;
                            wnode = self.visit(&RefCell::new(wnode));
                            vnode.lowlink = min(vnode.lowlink, wnode.lowlink);
                        } else {
                            let wnode = self.visited.get(&w).unwrap();
                            if wnode.on_stack {
                                vnode.lowlink = min(vnode.lowlink, wnode.index);
                            }
                        }
                    }
    
                    if vnode.lowlink == vnode.index {
                        let mut component = HashSet::new();
                        loop {
                            let mut wnode = self.dfs_stk.pop().unwrap();
                            wnode.on_stack = false;
                            self.visited.get_mut(&(wnode.vertex)).unwrap().on_stack = false;
                            component.insert(wnode.vertex);
                            // if wnode is vnode
                            if wnode.vertex == vnode.vertex {
                                break;
                            }
                        }
    
                        self.components.push(component);
                    }
                }
                _ => {}
            }
    
            *vnode
        }
    }
\end{lstlisting}

\subsubsection{Maximal modules}

The `maximal\_modules' function aims to find all maximal modules (subsets of nodes) in a graph that include a specific node `v'.

\begin{lstlisting}[language=Rust, style=rust, caption={Defining the maximal modules}, label={lst:rust-define-maximal-modules}, firstnumber=1]
    /// constructs M(g, v) according to the algorithm 3.1 of the article [1]
    ///
    /// # Arguments
    ///
    /// * 'g' - a two-structure
    /// * 'v' - a node of g
    ///
    /// # Returns
    ///
    /// a list of sets of nodes (modules)
    pub fn maximal_modules(g: &mut TwoStructure, v: u64) -> Vec<HashSet<u64>> {
        let mut init_module = g.nodes.clone();
        init_module.remove(&v);
        let mut partition = Vec::new();
        partition.push(init_module);
        let mut outsiders = Vec::new();
        let mut outsiders_h = HashSet::new();
        outsiders_h.insert(v);
        outsiders.push(outsiders_h);
        loop {
            if partition.is_empty() {
                panic!("Empty partition (please report)")
            }
            let mut module = HashSet::new();
            let mut index = 0;
            for i in 0..partition.len() {
                if outsiders.get(i).unwrap().len() as u64 > 0 {
                    // pick a module with outsiders
                    module = partition.get(i).unwrap().clone();
                    index = i;
                    break;
                }
            }
            if module.is_empty() {
                // if there is no module left having outsiders
                return partition;
            }

            // partition.remove(module)
            for i in 0..partition.len() {
                if *partition.get(i).unwrap() == module {
                    partition.remove(i);
                    break;
                }
            }

            let outsiders_i = outsiders.remove(index);
            // outsiders.remove(outsiders_i)


            // pick an arbitrary element
            let w = pickup(outsiders_i.clone());

            // list of sets of nodes
            let mut graph_alter = g.clone();
            let module_partition = module.clone();
            let r_modules = partition_module(&mut graph_alter, w, module_partition);
            for r_module in r_modules {
                let r_module_diff = r_module.clone();
                partition.push(r_module);
                // a = module - r_module
                let ha_alter = module.clone();
                let ha = ha_alter
                    .difference(&r_module_diff)
                    .collect::<HashSet<_>>();
                let mut a = HashSet::new();
                for &eha in ha {
                    a.insert(eha);
                }
                let mut b: HashSet<u64> = HashSet::new();
                if !outsiders_i.is_empty() {
                    for el in outsiders_i.iter() {
                        b.insert(*el);
                    }
                }
                b.remove(&w);
                outsiders.push(b);
            }
        }

        // this never happens
        // partition
    }
\end{lstlisting}

In arguments, we have:
\begin{itemize}
    \item `g': A reference to a `TwoStructure' representing the graph.
    \item `v': A node in the graph used as a reference for finding maximal modules.
\end{itemize}

and return a `Vec\textless HashSet\textless usize\textgreater\textgreater' where each `HashSet\textless usize\textgreater' represents a maximal module containing the node `v'.

\subsubsection{Partition module}

The `partition\_module' function aims to divide a given module (subset of nodes) into smaller submodules based on certain criteria related to the relationships between nodes in a graph.

\begin{lstlisting}[language=Rust, style=rust, caption={Defining the partition module}, label={lst:rust-define-partition module}, firstnumber=1]
    /// Partitions module in strong maximal modules
    ///
    /// # Arguments
    ///
    /// * 'g' - a two-structure
    /// * 'w' - the first node used for partitioning module
    /// * 'module' - a set of nodes of g to partition
    ///
    /// # Returns
    ///
    /// a list of sets of nodes
    pub fn partition_module(g: &mut TwoStructure, w: u64, module: HashSet<u64>) -> Vec<HashSet<u64>> {
        let colors_len = g.colors.len();

        // modules = [ [set() for c in g.colors] for c_ in g.colors ]
        let mut modules: Vec<Vec<HashSet<u64>>> = Vec::new();

        // is_empty = [ [True for c in g.colors] for c_ in g.colors ]
        let mut is_empty: Vec<Vec<bool>> = Vec::new();
        for _ in 0..colors_len {
            let mut vec_intern_modules_aux: Vec<HashSet<u64>> = Vec::new();
            let mut vec_intern_is_empty_aux: Vec<bool> = Vec::new();
            for _ in 0..colors_len {
                let set_inter_modules_aux: HashSet<u64> = HashSet::new();
                vec_intern_modules_aux.push(set_inter_modules_aux);
                vec_intern_is_empty_aux.push(false);
            }
            modules.push(vec_intern_modules_aux);
            is_empty.push(vec_intern_is_empty_aux);
        }

        for n in module {
            let c_w_n = g.color_of(w, n); // first color
            let c_n_w = g.color_of(n, w); // second color

            modules.get_mut(c_w_n as usize).unwrap().get_mut(c_n_w as usize).unwrap().insert(n);

            if *is_empty.get(c_w_n as usize)
                .unwrap()
                .get(c_n_w as usize)
                .unwrap()
            {
                is_empty.get_mut(c_w_n as usize)
                    .unwrap()
                    .push(false);
            }
        }

        let mut flattened: Vec<HashSet<u64>> = Vec::new();
        for m in modules {
            for s in m {
                if !s.is_empty() {
                    flattened.push(s);
                }
            }
        }
        flattened
    }
\end{lstlisting}

In Arguments, we have:
\begin{itemize}
    \item `g': A reference to a `TwoStructure' representing the graph.
    \item `w': A node in the graph used as a reference for partitioning.
    \item `module': A reference to a `HashSet<usize>' representing the module (subset of nodes) to be partitioned.
\end{itemize}

and return a `Vec\textless HashSet\textless usize\textgreater\textgreater' where each `HashSet\textless usize\textgreater' represents a submodule resulting from the partitioning process.


\subsubsection{Modular decomposition}

The `modular\_decomposition' function aims to decompose a graph into its modular components using a specified total order if provided.
It returns a hierarchical structure representing the modular decomposition.

\begin{lstlisting}[language=Rust, style=rust, caption={Defining the modular decomposition}, label={lst:rust-define-modular-decomposition}, firstnumber=1]
    fn modular_decomposition(g: &mut TwoStructure, total_order: Option<Vec<u64>>) -> TreeNode {
        if g.nodes.len() == 1 {
            let mut root = TreeNode::new();
            root.node_type = "LEAF".to_string();
            let iter = g.nodes.iter();
            let cloned_set: HashSet<u64> = iter.cloned().collect();
            root.node_id = Some(pickup(cloned_set));
            return root;
        }

        let mut total_order = total_order.unwrap_or_else(|| compute_total_order(g));
        total_order.retain(|&n| g.nodes.contains(&n));
        if total_order.is_empty() {
            total_order = compute_total_order(g);
        }

        let mut v = None;
        while let Some(node) = total_order.pop() {
            if g.nodes.contains(&node) {
                v = Some(node);
                break;
            }
        }
        let v = v.expect("No valid node found");

        let mods = maximal_modules(g, v);
        let mut qg = build_quotient(g, mods, v);
        let dg = build_distinction_graph(&mut qg, v);
        let mut cg = ComponentGraph::new(dg);

        let mut root = TreeNode::new();
        let mut u = &mut root;
        u.prime_graph = Some(qg.clone());

        while !cg.is_empty() {
            let mut w = TreeNode::new();
            w.node_type = "LEAF".to_string();
            w.node_id = Some(v);
            u.add_child(w);

            // get a mutable reference to the last child
            let u_ref = u;
            u = u_ref.children.last_mut().unwrap();

            let sinks = cg.remove_sinks();
            let mut fmodules: HashSet<u64> = HashSet::new();
            for sink in &sinks {
                for &mod_ in sink {
                    fmodules.insert(mod_);
                }
            }

            if sinks.len() == 1 && fmodules.len() > 1 {
                u.node_type = "PRIME".to_string();
                u.prime_graph = Some(qg.clone());
            } else {
                let iter = fmodules.iter();
                let p = pickup(iter.cloned().collect());
                let ntm = qg.node_to_module(p);
                match ntm {
                    Some(value) => {
                        let x = pickup((*value).clone());
                        if g.color_of(x, v) == g.color_of(v, x) {
                            u.node_type = "COMPLETE".to_string();
                            u.node_colors = vec![g.color_of(v, x)];
                        } else {
                            u.node_type = "LINEAR".to_string();
                            u.node_colors = vec![g.color_of(x, v), g.color_of(v, x)];
                        }
                    }
                    _ => {}
                }
            }

            for &fmod in &fmodules {
                let ntm = qg.node_to_module(fmod);
                match ntm {
                    Some(value) => {
                        let mut fg = g.slice((*value).clone());
                        let dtree = modular_decomposition(&mut fg, Some(total_order.clone()));
                        if ((u.node_type == "COMPLETE" && dtree.node_type == "COMPLETE")
                            && (u.node_colors == dtree.node_colors))
                            || ((u.node_type == "LINEAR" && dtree.node_type == "LINEAR")
                            && (u.node_colors == dtree.node_colors)) {
                            u.children.extend(dtree.children);
                        } else {
                            u.children.push(dtree);
                        }
                    }
                    _ => {}
                }
            }
        }
        root
    }
\end{lstlisting}

In arguments, we have:
\begin{itemize}
    \item `g': A mutable reference to a `TwoStructure' representing the graph to be decomposed.
    \item `total\_order': An optional vector of `u64' values specifying a total order of nodes.
    If provided, this order influences the decomposition process.
\end{itemize}

and return a `TreeNode' representing the hierarchical structure of the modular decomposition.

\hspace{4cm}


The code of all methods and functions defined in this stay in my GitHub repository~\cite{rustcode};


\section{Implementation in C++}\label{sec:implementation-in-c++}

C++ is a widely-used programming language known for its performance and control over system resources.
Here, he mainly helped me by giving me an approach to converting code from Python to Rust.
Firstly, it allowed me to properly type each variable in the Python code, then to differentiate between passing a variable by value or by reference when calling functions, and finally, it allowed me to clearly define the lifetime of the different variables declared in each function.

\subsection{Graph Representation in C++}\label{subsec:graph-representation-in-c++}

For the definition of the various graphs, we mainly use the same structures as those defined in the Rust code:

\subsubsection{2-structure}

\begin{lstlisting}[language=C++, style=cpp, caption={Defining the 2-Structure}, label={lst:cpp-define-2-structure}, firstnumber=1]
    class TwoStructure
    {
    public:
        /// the graph of a two structure
        set<unsigned int> nodes;
        /// list of sets, colors[i] contains the set of edges of color i
        vector< set< tuple<unsigned int, unsigned int> > > colors;
        /// only for quotient graph, contains the nodes of the graph if they are modules they
        vector< set<unsigned int> > modules;

        TwoStructure() {
            set< tuple<unsigned int, unsigned int> > s;
            colors.push_back(s);
        }
    };
\end{lstlisting}

\subsubsection{SCCNode}

\begin{lstlisting}[language=C++, style=cpp, caption={Defining the SCCNode}, label={lst:cpp-define-sccnode}, firstnumber=1]
    class SCCNode
    {
    public:
        unsigned int vertex;
        unsigned int index;
        unsigned int lowlink;
        bool on_stack;

        SCCNode(unsigned int v, unsigned int i) {
            vertex = v;
            index = i;
            lowlink = i;
            on_stack = false;
        }
    };
\end{lstlisting}

\subsubsection{SCC}

\begin{lstlisting}[language=C++, style=cpp, caption={Defining the SCC}, label={lst:cpp-define-scc}, firstnumber=1]
    template <class T>
    class SCC
    {
    public:
        T* graph;
        unsigned int index;
        vector<set<unsigned int>> components;
        unsigned int comp_index;
        vector<SCCNode*> dfs_stk;
        map<unsigned int, SCCNode*> visited;

        SCC(T* g) {
            graph = g;
            index = 0;
            comp_index = 0;
        }
    };
\end{lstlisting}

\subsubsection{MiniGraph}

\begin{lstlisting}[language=C++, style=cpp, caption={Defining the MiniGraph}, label={lst:cpp-define-mini-graph}, firstnumber=1]
    class MiniGraph
    {
    public:
        map<unsigned int, set<unsigned int>> dico;

        MiniGraph(map<unsigned int, set<unsigned int>> d) {
            dico = d;
        }
    };
\end{lstlisting}

\subsubsection{DGraph}

\begin{lstlisting}[language=C++, style=cpp, caption={Defining the DGraph with C++}, label={lst:cpp-define-d-graph}, firstnumber=1]
    class MiniGraph
    {
    public:
        map<unsigned int, set<unsigned int>> dico;

        MiniGraph(map<unsigned int, set<unsigned int>> d) {
            dico = d;
        }
    };
\end{lstlisting}

\subsubsection{ComponentGraph}

\begin{lstlisting}[language=C++, style=cpp, caption={Defining the ComponentGraph}, label={lst:cpp-define-component-graph}, firstnumber=1]
    class ComponentGraph {
    public:
        DGraph graph;
        vector<set<unsigned int>> components;
        vector<set<unsigned int>> modules;

        ComponentGraph(DGraph g) {
           graph = g;
           components = strongly_connected_components(graph);
           modules = graph.modules;
        }
    };
\end{lstlisting}

\subsubsection{TreeNode}

\begin{lstlisting}[language=C++, style=cpp, caption={Defining the TreeNode}, label={lst:cpp-define-tree-node}, firstnumber=1]
    class TreeNode {
    public:
        string node_type;
        unsigned int node_colors;
        vector<TreeNode> children;
        unsigned int node_id;
        TwoStructure prime_graph;
    };
\end{lstlisting}

\subsection{Functions in C++}\label{subsec:functions-in-c++}

\subsubsection{Maximal modules}

\begin{lstlisting}[language=C++, style=cpp, caption={Defining Maximale Module}, label={lst:cpp-define-maximal-module}, firstnumber=1]
    vector<set<unsigned int>> maximal_modules(TwoStructure *g, unsigned int v) {
        set<unsigned int> init_module(g->nodes);
        init_module.erase(v);
        vector<set<unsigned int>> partition({init_module});
        vector<set<unsigned int>> outsiders({set<unsigned int>({v})});
        set<unsigned int> module;
        int cnt = 1;
        while (true) {
           if (partition.empty()) {
              throw "Empty partition (please report)";
           }
           // module = None
           module.clear();
           // bool none_module = true;
           unsigned int index = 0;
           for (unsigned int i = 0; i < partition.size(); i++) {
              if (!outsiders.at(i).empty()) {
                 // pick a module with outsiders
                 module = partition.at(i); // that's a set of nodes
                 // none_module = false;
                 index = i;
                 break;
              }
           }

           if (module.empty()) {
              // if there is no module left having outsiders
              return partition;
           }
           // partition.remove(module)
           auto pos_module =
              find(partition.begin(), partition.end(), module);
           if (pos_module != partition.end()) {
              partition.erase(pos_module);
           }
           set<unsigned int> outsiders_i = outsiders.at(index);
           // outsiders.remove(outsiders_i)
           auto pos_outsiders_i =
              find(outsiders.begin(), outsiders.end(), outsiders_i);
           if (pos_outsiders_i != outsiders.end()) {
              outsiders.erase(pos_outsiders_i);
           }
           unsigned int w = pickup(outsiders_i); // pick an arbitrary element

           vector<set<unsigned int>> rmodules =
              partition_module(g, w, module); // list of sets of nodes
           for (auto & rmodule : rmodules) {
              partition.push_back(rmodule);
              // a = module - rmodule
              set<unsigned int> a(module);
              for (unsigned int rm : rmodule) {
                 a.erase(rm);
              }
              /*
              if outsiders_i != None:
                 b = a.union(outsiders_i)
              else:
                 b = a
              */
              set<unsigned int> b(a);
              if (!outsiders_i.empty()) {
                 for (unsigned int o : outsiders_i) {
                    b.insert(o);
                 }
              }
              b.erase(w);
              outsiders.push_back(b);
           }
        }
        return partition; // this never happens
    }

\end{lstlisting}

\subsubsection{Partition module}

\begin{lstlisting}[language=C++, style=cpp, caption={Defining the Partition Module}, label={lst:cpp-define-partition-module}, firstnumber=1]
    vector<set<unsigned int>> partition_module(TwoStructure *g, unsigned int w, const set<unsigned int>& module) {
        unsigned int colors_len = g->colors.size();
        unsigned int i, j;

        // modules = [ [set() for c in g.colors] for c_ in g.colors ]
        vector<vector<set<unsigned int>>> modules(
           colors_len, vector<set<unsigned int>>(colors_len, set<unsigned int>()));

        // is_empty = [ [True for c in g.colors] for c_ in g.colors ]
        vector<vector<bool>> is_empty(colors_len, vector<bool>(colors_len, true));

        for (unsigned int n : module) {
           unsigned int c_w_n = g->color_of(w, n); // first color
           unsigned int c_n_w = g->color_of(n, w); // second color

           modules.at(c_w_n).at(c_n_w).insert(n);

           if (is_empty.at(c_w_n).at(c_n_w)) {
              is_empty.at(c_w_n).at(c_n_w) = false;
           }
        }

        // remove empty set
        vector<set<unsigned int>> flattened;
        for (auto & module : modules) {
           for (auto & s : module) {
              if (!s.empty()) {
                 flattened.push_back(s);
              }
           }
        }
        return flattened;
    }
\end{lstlisting}

\subsubsection{Modular decomposition}

\begin{lstlisting}[language=C++, style=cpp, caption={Defining the Modular Decomposition}, label={lst:cpp-define-modular-decomposition}, firstnumber=1]
    TreeNode modular_decomposition(TwoStructure g) {
        if (g.nodes.size() <= 1) {
           TreeNode root;
           root.node_type = "LEAF";
           root.node_id = pickup(g.nodes);
           return root;
        }


        // select an arbitrary node of g
        unsigned int v = pickup(g.nodes);

        // Compute the maximal modules of g wrt. v
        vector<set<unsigned int>> mods = maximal_modules(&g, v);

        // Compute the quotient structure
        TwoStructure qg = build_quotient(g, &mods, v);


        // Compute the 'distinguish' (?) graph wrt. the image of v
        DGraph dg = build_distinction_graph(qg, &v);

        // Compute the component graph
        ComponentGraph cg(dg);


        TreeNode root;
        TreeNode *u = &root;
        // u->quotient = qg;
        while (!cg.is_empty()) {
           TreeNode w;
           u->add_child(w);
           w.node_type = "LEAF";
           w.node_id = v; // line missing //

           vector<set<unsigned int>> sinks = cg.remove_sinks();
           set<unsigned int> fmodules;

           for (auto & sink : sinks) {
              for (unsigned int mod : sink) {
                 fmodules.insert(mod);
              }
           }

           if (sinks.size() == 1 && fmodules.size() > 1) {  // Set the type of node u
              u->node_type = "PRIME"; // one complete connected component
              u->prime_graph = qg;
           } else {
                cout << "g.nodes().size() = " << fmodules.size() << endl;
              unsigned int x = pickup(qg.node_to_module(pickup(fmodules)));
              if (g.color_of(x, v) == g.color_of(v, x)) {
                 u->node_type = "COMPLETE"; // sink is one
                 u->node_colors = g.color_of(v, x);
              } else {
                 u->node_type = "LINEAR"; // sinks maybe one or more
                 u->node_colors = g.color_of(x, v);
                 // dans le code python, on a u.node_colors = {g.color_of(c, v),
                 // g.color_of(v, x)} u->node_colors est un entier représentant
                 // une couleur ou un ensemble ???
              }
           }

           // Recursive calls
           // fmodule == fmod
           for (unsigned int fmodule : fmodules) {
              TwoStructure fg = g.slice(qg.node_to_module(fmodule));

              TreeNode dtree = modular_decomposition(fg);

              if (((u->node_type == "COMPLETE" && dtree.node_type == "COMPLETE") && (u->node_colors == dtree.node_colors))
                        || (u->node_type == "LINEAR" && dtree.node_type == "LINEAR") && (u->node_colors == dtree.node_colors)) {
                 for (auto & child : dtree.children) {
                    u->children.push_back(child);
                 }
              } else {
                 u->children.push_back(dtree);
              }
           }
           u = &w;
        }

        return root;
    }
\end{lstlisting}


