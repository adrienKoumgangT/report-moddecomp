%! Author = adrienkoumgangtegantchouang
%! Date = 01/10/24

\chapter{Algorithm of modular decomposition}\label{ch:algorithm-of-modular-decomposition}

Modular decomposition is a powerful technique used to simplify and analyze the structure of graphs by breaking them down into modules.
One of the notable algorithms for achieving this is the algorithm developed by Ehrenfeucht et al.
This section delves into the details of the algorithm, its steps, and its application in graph theory.

\section{Introduction of Partition Tree Family}\label{sec:introduction-of-partition-tree-family}

In the article ``An $O(n^2)$ Divide-and-Conquer Algorithm for the Prime Tree Decomposition of Two-Structure and Modular Decomposition of Graphs article\cite{PTDMD}", Ehrenfeucht and co work directly on two-structured graphs, and the application of their algorithms can easily be extended to other graph.
A two-structure is the generalization that allows an arbitrary coloring of this set.
A two-structure may thus be viewed as an edge-colored complete directed graph.
Graphs are a special case of two-structures, so all theorems and algorithms presented in this chapter for two-structures are immediately applicable to graphs.


If $g$ is a two-structure, $dom(g)$ denotes its nodes.
The substructure induced in $g$ by $X \subseteq dom(g)$, denoted $g \mid X$, is the subgraph of $g$ induced by $X$, where the edges in $g \mid X$ have the same color as they do in $g$.
A node $x$ distinguishes nodes $y$ and $z$ if $(x, y)$ and $(x, z)$ are different colors,or $(y, x)$ and $(z, x)$ are different colors.
A module is a set $X \subseteq dom(g)$ such that no $x \in dom(g) - X$ distinguishes any two members of $X$.
It is easily seen that if two modules are disjoint, all edges from one of the modules to the other are the same color.
If $R$ is a partition of the nodes of $g$, a system of representatives from $R$ is a set consisting of one node from each member of $R$.
If each member of $R$ is a module of $g$, then every system of representatives induces the same substructure.
This substructure is denoted $g / R$, and completely specifies the colors of all edges that are not internal to a member of $R$.
The operation may be performed recursively on each member of $R$ by partitioning it into still smaller modules.
The result is a compact, hierarchical decomposition of the entire two-structure.

The \textit{prime tree family} of a two-structure is such a hierarchical decomposition, and it is unique.
It is given by the family of \textit{prime modules} : $\{X : X$ is a nonempty module of $g$, and for any module $Y$, either $Y \subseteq X, X \subseteq Y$, or $X \cap Y = \emptyset\}$.
If $\mid X \mid > 1$, the maximal-cardinality members of the prime tree family that are proper subsets of $\mid X \mid$, denoted $children_g(X)$, are a partition of $X$.
Thus $(g \mid X) / children_g(X)$, given for all $X$ in the prime tree family, completely represents the original two-structure.
As is described below, this family represents in a simple way all modules in the two-structure, not just those that are members of the family.
The number of modules may be exponential in $n$, but this representation takes space that is linear in $n$.

The prime tree family for the case of graphs is also known as the modular decomposition\cite{IMD}.


\section{Preliminaries}\label{sec:preliminaries}

Let $G$ be a directed graph.
The \textit{component graph} for $G$ has one node for each strongly connected component.
If $X$ and $Y$ are two strongly connected components of $G$, then $(X,\, Y)$ is an edge in the component graph if there exist $x \in X$ and $y \in Y$ such that $(x,\, y)$ is an edge of $G$.

Let $X$ and $Y$ be finite sets.
$X$ and $Y$ are overlapping if and only if $X - Y$, $ X \cap Y$, and $Y - X$ are all nonempty.
A two-edge over $X$ is an ordered pair $(x,\, y)$ such that $x \neq y$ and $x, y \in X$.
In this section, the term edge refers to a two-edge.
The set $E_2(X)$ is the set of all possible edges over $X$, and a two-structure on domain $X$ is a coloring of $E_2(X)$.
An edge $(x,\, y)$ is symmetric if $(y,\, x)$ is the same color.
A two-structure is symmetric if all of its edges are symmetric.
$X$ is a singleton set if $\mid X \mid = 1$.
The nodes of a two-structure $g$ are known as its \textit{domain}, denoted $dom(g)$, and $g$ is a coloring of $E_2(dom(g))$.
Obviously, $dom(g)$, $\emptyset$, and the singleton subsets of $dom(g)$ are modules of $g$; these are the trivial modules of $g$.
In this section, modules will be assumed to be nonempty, except when otherwise stated.

Let $g$ be a two-structure, and let $R$ be a partition of $dom(g)$ such that every member of $R$ is a module of $g$.
Let $g'$ be the two-structure $g / R$.
If $X \subseteq dom(g')$, its inverse image in $g$ is the union of the members of $R$ that correspond to members of $X$.
$X$ is the image in $g'$ of its inverse image.

\begin{mytheo}
    \label{thm:2-1}
    Let $g$ be a two-structure, and let $X$ and $Y$ be overlapping modules of $g$.
    Then $X \cap Y$, $X \cup Y$, and $X - Y$ are modules of $g$.
\end{mytheo}

\begin{mytheo}
    \label{thm:2-2}
    Let $g$ be a two-structure, and let $R$ be a partition of $dom(g)$ into modules.
    \begin{enumerate}
        \item If $X$ is a module in $g$ that has an image $Y$ in $g / R$, then Y is a module in $g / R$.
        Moreover, if $X$ is a prime module in $g$, then $Y$ is a prime module in $g / R$.
        \item If $Y$ is a module in $g / R$, then its inverse image, $X$, is a module in $g$.
        Moreover, if the members of $R$ are prime modules in $g$ and $Y$ is a prime module in $g / R$, $X$ is a prime module in $g$.
    \end{enumerate}
\end{mytheo}

\begin{mylem}
    \label{lem:2-3}
    Let $g$ be a two-structure, and let $X$ be a module of $g$.
    Then the modules of $g \mid X$ are those modules of $g$ that are subsets of $X$.
    Moreover, the prime modules of $g \mid X$ that are proper subsets of $X$ are the prime modules of $g$ that are proper subsets of $X$.
\end{mylem}


\begin{mydef}
    \label{def:2-4}
    Let $g$ be a two-structure.
    \begin{enumerate}
        \item $g$ is \textit{primitive} if and only if it contains at least three nodes and all modules in $g$ are trivial.
        \item $g$ is \textit{complete} if and only if all of its edges are the same color.
        \item $g$ is \textit{linear} if and only if there exists a linear order $(x_1, \dots, x_{\mid dom(g) \mid})$ of the elements of $dom(g)$ such that the edges $\{(x_i,\, x_j): i < j\}$ are the same color, the edges $\{(x_j,\, x_i): i < j\}$ are the same color, and the colors of these two sets are different.
    \end{enumerate}
\end{mydef}

It is clear that if $g$ is complete, its modules are all subsets of $dom(g)$, and if it is linear, then its modules are the consecutive sets (sets of the form $\{x_i, x_{i+1}, \dots, x_j\}$) in the linear order on its nodes.

\begin{mytheo}
    \label{thm:2-5}
    .\\
    \begin{enumerate}
        \item If $X$ is a prime module of $g$ and $\mid X \mid > 1$, $children_g(X)$ is a partition of $X$;
        \item Each module of a two-structure $g$ is a union of siblings in the prime tree family of $g$;
        \item If $X$ is a prime module of $g$ such that $\mid X \mid > 1$, $X$ is of one of the following types:
            \begin{itemize}
                \item \textbf{primitive}: no union of more that one and less than all of its children is a module of $g$;
                \item \textbf{complete}: every union of a subfamily of its children is a module of $g$;
                \item \textbf{linear}: there exists a linear ordering on its children such that the union of a subfamily is a module of $g$ if and if the subfamily is consecutive in that ordering.
            \end{itemize}
    \end{enumerate}
\end{mytheo}

\begin{mycor}
    \label{cor:2}
    If $X$ is a member of the prime tree family, then $(g \mid X) / children_g(X)$ is either \textit{primitive}, \textit{linear}, or \textit{complete}.
\end{mycor}

By Theorem~\ref{thm:2-5}, we way represent all modules of $g$ as follows.
Create one node to represent each prime module.
If $x$ represents a prime module $X$, establish an edge from $x$ to $y$ whenever $y$ represents a member of $children_g(X)$.
We will call this data structure $ptf(g)$.
In $ptf(g)$, the leaf descendants of node $x$ give the prime module $X$ it represents.
In addition, if $\mid X \mid > 1$, one may label $x$ as primitive, linear or complete, and supply the appropriate ordering of its children if it is linear.
If $x$ is complete, its color is the color of the edges connecting its children.
If $x$ is linear, its colors are the two colors of edges connecting its chidlren.

We seek to compute $ptf(g)$.
For ease of notation, when $x$ is a node of $ptf(g)$ that represents a prime module $X$, we will view $x$ and $X$ as synonymous.


\section{The Algorithm}\label{sec:the-algorithm}

In this section, we give an algorithm that computes $ptf(g)$ for symmetric two-structures.
Since undirected graphs are symmetric two-structures, they are covered.
Later, we will give the minor modification needed when the two-structure may have asymmetric edges.
We will assume that the two-structure is given as an adjacency array, and that if the two-structure has $k$ edge colors, they are given in the adjacency array as integers from zero to $k-1$.

$G(g, v)$ is trivially computed in $O(n^2)$ time. $M(g, v)$ is a partition of $dom(g) - \{v\}$, and it may be computed with Algorithm~\ref{alg:compute}, variants of which have appeared repeatedly in related contexts.
The prime tree family of an arbitrary two-structure is then computed with Algorithm~\ref{alg:ptf}.


\begin{mydef}
    \label{def:3-1}
    Let $u$ be a node of $dom(g)$; $G(g, u)$ denotes a graph whose nodes are given by $dom(g) - (0)$.
    There is an edge in $G(g, v)$ from node $x$ to node $y$ if $x$ distinguishes $y$ and $v$ in $g$.
    $M(g,u)$ denotes the family of maximal modules of $g$ that do not contain $v$.
    That is, $X \in M(g, 0)$ iff $X$ is a module, and for every module $Y$ such that $X \subset Y$, $Y$ contains $v$.
\end{mydef}


\begin{algorithm}[H]
    \label{alg:compute}
    \SetAlgoLined
    \caption{Compute M(g, v)}
    Maintain a family $L$ of partition classes, and for each partition class $S$, maintain a set $Z(S)$ of "unprocessed outsiders".\\
    Initially, there is one partition class $S = \text{dom}(g) - \{u\}$ in $L$, with $Z(S) = \{u\}$.

    \While{$L$ contains a class $S$ such that $Z(S)$ is nonempty}{
        Remove $S$ from $L$\;
        Let $w$ be an arbitrary member of $Z(S)$\;
        Partition $S$ into the maximal subsets that are not distinguished by $w$\;
        \For{each resulting subset $W$}{
            Make $W$ a member of $L$\;
            Let $Z(W) = (S - W) \cup (Z(S) - \{w\})$\;
        }
    }
\end{algorithm}


\begin{algorithm}[H]
    \caption{ptf(g)}
    \label{alg:ptf}
    Select a node $u$ of $g$ and compute $M(g, u)$ using Algorithm \ref{alg:compute}\;
    Let $g' = g / (M(g, u) \cup \{\{u\}\})$\;
    Let $u'$ be the image of $u$ in $g'$\;
    Let $G' = G(g', u')$\;
    Let $G''$ be the component graph of $G'$\;
    Create a tree node $t$\;
    $u := t$\;
    \While{$G''$ is not empty}{
        Create a tree node $w$ and make it a child of $u$\;
        Remove a sink from $G''$; let $F$ be the corresponding members of $M(g, u)$\;
        \eIf{$\mid F \mid > 1$}{
            $u$ is primitive\;
        }{
            $u$ is complete\;
        }
        \ForEach{member $X$ of $F$}{
            Compute $ptf(g \mid X)$ recursively\;
            \eIf{$u$ and the root of $ptf(g \mid X)$ are both complete and have the same color}{
                Make the children of $ptf(g \mid X)$ be children of $u$\;
            }{
                Make $ptf(g \mid X)$ be a child of $u$\;
            }
        }
        $u := w$\;
    }
    \Return $t$\;
\end{algorithm}


\section{Correctness and Time bound}\label{sec:correctness-and-time-bound}

\subsection{Algorithm \ref{alg:compute}}\label{subsec:algorithm-compute}

\begin{mylem}
    \label{lem:lemma-4-1}
    Let $g$ be a two-structure, and let $v \in dom(g)$.
    Then $M(g, v) \cup \{\{v\}\}$ is a partition of $dom(g)$.
\end{mylem}

\textbf{Proof.}
If a node $w \neq v$ is not in any member of $M(g, v)$, then there is no module containing $w$ but excluding $v$, contradicting the fact that $\{w\}$ is such a module.
Suppose $w$ is in two members, $X$ and $Y$, of $M(g, v)$.
The union of $X$ and $Y$ is a module that does not contain $v$, since one contains the other or else they overlap, in which case their union is a module by Theorem~\ref{thm:2-1}.
They cannot both be maximal modules not containing $v$, contradicting membership of both of them in $M(g, v)$.
Thus, each node of $dom(g) - \{v\}$ is a member of exactly one set in $M(g, v)$.
$M(g, v)$ is a partition of $dom(g) - \{v\}$, and the lemma follows.


\begin{mylem}
    \label{lem:4-2}
    Algorithm~\ref{alg:compute} computes $M(g, v)$.
\end{mylem}

\textbf{Proof.}
The algorithm clearly maintains the following invariant: For each partition class $S$, a member $x$ of $dom(g) - S$ may distinguish members of $S$ only if it is a member of the outside list for $S$.
The algorithm terminates when every outsider list is empty, hence, when each partition class in a module.
Conversely, if a module of $g$ does not contain $v$, it is a subset of the initial member of $L$, and its members cannot be split into different partition classes by any outsider.
Thus, its memebers are all in the same final partition class.
Since each maximal module that does not contain $v$ is a final partition class, the final partition classes must be the maximal modules that do not contain $v$, by Lemma~\ref{lem:lemma-4-1}.

\begin{mydef}
    \label{def:4-3}
    Let $g$ be a two-structure, and let $R$ be a partition of $dom(g)$.
    The edges that are exposed by $R$ are the set $\{(u, v): u, v \in dom(g)$ and $u$ and $v$ are not in the same partition class of $R\}$
\end{mydef}

\begin{mylem}
    \label{lem:4-4}
    Let $g$ be a two-structure, let $v \in dom(g)$, and let $k$ be the number of edges exposed by $M(g, v) \cup \{\{v\}\}$.
    The number of operations required by Algorithm~\ref{alg:compute} is $O(k)$.
\end{mylem}

\textbf{Proof.}
Maintain all sets $L$ and their outsider sets as linked lists.
A set $S$ may be partitioned with an outsider $w$ in $O(\mid S \mid)$ time as follows.
Use a two-phase bucket sort.
In the first phase, bucket sort the members of $S$ according to the color of the edges from $w$ to those members.
In the second phase, bucket sort each of the resulting nonempty buckets according to the colors of the edges from those members to $w$.
Charge the cost of the partition to the edges connecting w and the members of $S$.
Charge the cost of having originally inserted $w$ on the outsider set for $S$ to one of these edges.
Since w is not on any of the new outsider sets, these edges do not receive any further charges in later iterations, so they are each charged constant cost over all executions.
Only exposed edges receive charges.
Thus, the costs of performing all partitions and maintaining all lists is $O(k)$.

\subsection{Algorithm \ref{alg:ptf}}\label{subsec:algorithm-ptf(g)}

\begin{mylem}
    \label{lem:5-1}
    Let $g$ be a two-structure, and let $v \in dom(g)$, where $\mid dom(g) \mid > 1$.
    Let $U \neq \{v\}$ be a proper ancestor of $\{v\}$ in $ptf(g)$, and let $W$ be $U$'s child that contains $v$:
    \begin{enumerate}
        \item If $(g \mid U) / children_g(U)$ is primitive, each of $U$'s children, other than $W$, is a member of $M(g, v)$.
        \item If $(g \mid U) / children_g(U)$ is complete, the union of all of $U$'s children except $W$ is a member of $M(g, v)$.
        \item If $(g \mid U) / children_g(U)$ is linear, the union of children of $U$ that are before $W$ in the linear order is a member of $M(g, v)$ or empty.
        The same is true of any children after $W$.
    \end{enumerate}

    There are no members of $M(g, v)$ other than those given by the above rule when it is applied to all ancestors of $v$.
\end{mylem}

\textbf{Proof.}
Follows from Theorem~\ref{thm:2-5}

\begin{mytheo}
    \label{thm:5-2}
    Let $g' = g / (M(g, v) \cup \{\{v\}\})$.
    If $g$ is symmetric, then the modules of $g'$ that contain the image of $\{v\}$ are prime in $g'$, and their inverse images in $g$ give the ancestors of $\{v\}$ in the prime tree family of $g$.
\end{mytheo}

\textbf{Proof.}
Suppose $U$ is a proper ancestor of $(v)$ in the prime tree family of $g$ and $W$ is its child that contains $v$.
By Lemma~\ref{lem:5-1} , $U$ and $W$ are the inverse images of sets of nodes of $g'$.
By Theorem~\ref{thm:2-2}, their images in $g'$ are prime in $g'$.
If $g$ is symmetric, $U$ is complete or primitive, so by Theorem~\ref{thm:2-2} and Lemma~\ref{lem:5-1}, the image of $U$ is the smallest module of $g'$ that contains the image of $W$.
Applying this argument to all ancestors $U$ of $(v)$ shows that all modules of $g'$ that contain the image of $(u)$ are the images
of the ancestors of $(v)$ in the prime tree family of $g$.

\begin{mylem}
    \label{lem:5-3}
    Let $g$ be a two-structure, and let $X$ be the set of nodes reachable from node $x$ in $G(g, v)$.
    The set $dom(g) - X$ is the largest module of $g$ that contains $u$ and excludes $x$.
\end{mylem}

\textbf{Proof.}
If there is an edge from $x$ to $y$ in $G(g, v)$ then $(x, y)$ and $(x, v)$ are different colors or $(y, x)$ and $(v, x)$ are different colors, which means that any module containing u and excluding x must also exclude y.
If $y$ is excluded from the module, then the same argument shows that any node reachable from $y$ on a single edge must also be excluded from the module.
Transitively, every node reachable from $x$ on any path must be excluded from the module.
To see that $dom(g) - X$ is, in fact, a module, we observe that the nodes in $dom(g) - X$ are not reachable in $G(g, u)$ from any node in $X$, which means that for any node $z$ in $X$ and any node $u$ in $dom(g) - $X, $(z, u)$ and $(z, v)$ are the same color and $(u, z)$ and $(v,z)$ are the same color.
Transitively, for any two nodes u and w in $dom(g) - X$, $(z, u)$ and $(z, w)$ are the same color and $(u, z)$ and $(w, z)$ are the same color.
Thus, $dom(g) - X$ is a module.

\begin{mycor}
    \label{cor:}
    Let $X$ be a set corresponding to a sink in the component graph of $G(g, v)$.
    Then $dom(g) - X$ is a maximal-cardinality module that contains $v$ and that is not equal to $dom(g)$.
\end{mycor}

\begin{mypro}
    \label{prop:5-4}
    Let $g$ be a two-structure, let $v \in dom(g)$, and let $X \subseteq dom(g) - \{v\}$.
    Let $g' = g \mid (dom(g) - X)$.
    Then $G(g', v)$ is the sub-graph induced in $G(g, v)$ by $dom(g) - (X \cup \{v\})$.
\end{mypro}

\begin{mytheo}
    \label{thm:5-5}
    Let $g$ be a two-structure, and let $W$ be a child of $U$ in $ptf(g)$.
    If $U$ and $W$ are both complete, the color of the edges connecting children of $W$ is different from the color of the edges connecting children of $U$.
    If $U$ and $W$ are both linear, the pair of colors of edges connecting children of $W$ is different from the pair of colors connecting children of $U$.
\end{mytheo}

The correctness of Algorithm~\ref{alg:ptf} now follows.
Algorithm~\ref{alg:ptf} clearly returns the correct result whenever $\mid dom(g) \mid = 1$.
Let the inductive hypothesis be that it returns the correct result whenever $\mid dom(g) \mid < k$.
Suppose that $\mid dom(g) \mid = k$.
Let $W$ be the child of $dom(g)$ in $ptf(g)$ that contains $v$, and let $g' = g/(M(g,u) \cup \{\{v\}\}$.
By Theorem~\ref{thm:5-2}, there is a unique maximal module of $g'$ that is a proper subset of $dom(g')$ and that contains the image of $\{v\}$.
Thus, by the corollary to Lemma~\ref{lem:5-3}, there is a unique sink in the component graph for $G(g',u')$.
By Theorem~\ref{thm:5-2}, $W = dom(g') - \cup F$.
The characterization of $U$ as primitive or complete is correct, by Lemma~\ref{lem:5-1}.
By the inductive hypothesis, the recursive call produces the correct tree for $g \mid X$, so by Lemma~\ref{lem:2-3} and Theorem~\ref{thm:5-5}, the main routine correctly attaches all siblings of $W$ and their descendants as children of $dom(g)$.
By Proposition~\ref{prop:5-4} and Lemma~\ref{lem:2-3}, subsequent iterations of the main loop of Algorithm~\ref{alg:compute} are computationally equivalent to a recursive call to Algorithm~\ref{alg:compute} on $g \mid W$.
By the inductive hypothesis, they produce the subtree of $ptf(g)$ rooted at $W$.
Thus, Algorithm~\ref{alg:compute} produces the correct result when $\mid dom(g) \mid \leq k$.
Inductively, it produces the correct result for all symmetric two-structures.

For the time bound, let $k$ be the number of edges exposed by $M(g, v) \cup \{\{v\}\}$.
If the two-structure has only one node, we may charge the cost of returning the trivial decomposition to the node.
Otherwise, by Lemma~\ref{lem:4-4}, computing$ M(g, v)$ requires $O(k)$ time, so we may charge this cost to the exposed edges.
Let $g' = g/(M(g, v) \cup \{\{v\}\})$.
The number of nodes of g' is $O(k^{1/2})$.
Computing $G(g',v')$ and the component graph of $G(g', v')$ takes $O(k)$ time, as does the cost of identifying and removing sinks from the component graph.
These costs may be charged to the exposed edges at constant cost per edge.
In recursive calls, we use the same charging scheme on edges exposed in those calls.
Each recursive call generated from the main procedure occurs on the substructure induced by a single member of $M(g, v)$, and thus, the edges charged in one recursive call are disjoint from those charged in either the main procedure or any other recursive call.
It follows that all costs are charged to the edges at constant time per edge, giving the $O(n^2)$ time bound on the algorithm.



\section{The generalization of Algorithm \ref{alg:ptf} for arbitrary Two-Structure}\label{sec:the-generalization-of-algorithmref{alg:ptf}-for-arbitrary-two-structure}

Algorithm~\ref{alg:3} gives the generalization of Algorithm~\ref{alg:ptf} for arbitrary two-structures.
For the correctness, note that when asymmetric edges are allowed, case 3 of Lemma\ref{lem:5-1} can no longer be excluded.
Because of this, Theorem~\ref{thm:5-2} is no longer true.
Instead, we make use of the following.

\begin{mylem}
    \label{lem:6-1}
    Let $g$ be a two-structure, let $v$ be a node of $g$, and let $W$ be the child of $dom(g)$ that contains $v$.
    Let $g' = g / (M(g, v) \cup \{\{v\}\})$, let $v'$ be the image of $v$ in $g'$, and let $W'$ be the image of $W$.
    Let $G''$ be the component graph of $G(g', v')$, and let $X'$ be the nodes of $g'$ that correspond to the sinks of $G''$.
    Then $W' = dom(g') - X'$.
\end{mylem}

\textbf{Proof.}
By Lemma~\ref{lem:5-1}, $W$ has an image in $g'$, and by Theorem~\ref{thm:2-2}, $W'$ is prime in $g'$.
By Theorem~\ref{thm:2-2}, if the root of $ptf(g)$ is primitive, $W'$ is the unique maximal module of $g'$ that contains $v'$ and that is not equal to $dom(g')$.
If the root of $ptf(g)$ is complete, then by Theorem~\ref{thm:2-2} and Lemma~\ref{lem:5-1}, $W'$ has only one sibling in $ptf(g')$, so it is again the unique maximal module of $g'$ that contains $v'$ and that is not equal to $dom(g')$.
In either case, Lemma~\ref{lem:6-1} is true by the corollary to Lemma~\ref{lem:5-3}.
If the root of $ptf(g)$ is linear, then by Theorem~\ref{thm:2-2} and Lemma~\ref{lem:5-1}, $ W'$ has one or two siblings in $ptf(g')$.
If it has one sibling, it is the unique maximal module of $g'$ that contains $v'$ and that is not equal to $dom(g')$, so again, the lemma is true.
Otherwise, the union of $W'$ and either of these siblings is thus a maximal module of $g'$ that contains v and is not equal to $dom(g)$.
Thus, each of these siblings is a sink of $G''$, by the corollary to Lemma~\ref{lem:5-3}.
These are the only sinks of $G''$, since $W'$ is prime in $g'$.
Thus, the lemma is true in this case also.


The proof of correctness of Algorithm~\ref{alg:3} is similar to the one for Algorithm~\ref{alg:ptf}: Algorithm~\ref{alg:3} clearly returns the correct result whenever $\mid dom(g) \mid = 1$.
Let the inductive hypothesis be that it returns the correct result whenever $\mid dom(g) \mid < k$.
Suppose $\mid dom(g) \mid = k$.
Let $W$ be the child of $dom(g)$ in $ptf(g)$ that contains 4, and let $g' = g/(M(g, u) \cup \{\{v\}\})$.
Let $X'$ be the nodes of g' that correspond to the sinks of $G''$.
By Lemma~\ref{lem:6-1}, $W$ is the inverse image of $dom(g') - X'$.
The characterization of u as primitive is correct, by Lemma~\ref{lem:5-1}.
If there are two sinks in $G''$, $u$ is linear.

Otherwise, it is linear or complete.
Which case holds may be found by examining whether an edge such as $(v, x)$ that connects its children is symmetric.
By the inductive hypothesis, the recursive call produces the correct tree for $g \mid X$, for each $X$ e $M(g,v)$, so by Lemma~\ref{lem:2-3} and Theorem~\ref{thm:5-5}, the main routine correctly attaches all siblings of $W$ and their descendants as children of $dom(g)$.
By Proposition~\ref{prop:5-4} and Lemma~\ref{lem:2-3}, subsequent iterations of the main loop of Algorithm~\ref{alg:3} are computationally equivalent to a recursive call to Algorithm~\ref{alg:3} on $g \ mid W$.
By the inductive hypothesis, they produce the subtree of $ptf(g)$ rooted at $W$.
Thus, Algorithm~\ref{alg:3} produces the correct result when $\mid dom(g) \mid \leq k$.
Inductively, it produces the correct result for all two-structures.

The proof of the $O(n^2)$ time bound for Algorithm~\ref{alg:3} is identical to the one for Algorithm~\ref{alg:ptf}.

\begin{algorithm}[H]
    \caption{ptf(g)}
    \label{alg:3}
    Select an arbitrary node $v$ of $g$ and compute $M(g, v)$\;
    Let $g' = g / (M(g, v) \cup \{\{v\}\})$\;
    Let $(v')$ be the image of $\{v\}$ in $g'$\;
    Let $G' = G(g', u')$\;
    Let $G''$ be the component graph of $G'$\;
    Create a tree node $t$\;
    Let $u := t$\;
    \While{$G''$ is not empty}{
        Create a tree node $w$ and make it a child of $u$\;
        Remove all sinks from $G''$; let $F$ be the corresponding members of $M(g, v)$\;
        \eIf{only one sink was removed from $G''$ and $\mid F \mid > 1$}{
            $u$ is primitive\;
        }{
            Select an arbitrary node $x$ from a member of $F$\;
            \eIf{$(v, x)$ and $(x, v)$ are the same color}{
                $u$ is complete\;
            }{
                $u$ is linear\;
            }
        }
        \ForEach{member $X$ of $F$}{
            Compute $ptf(g \mid X)$ recursively\;
            \eIf{$u$ and the root of $ptf(g \mid X)$ are both complete and the same color or both linear and the same color}{
                Make the children of $ptf(g \mid X)$ be children of $u$\;
            }{
                Make $ptf(g \mid X)$ be a child of $u$\;
            }
        }
        $u := w$\;
    }
    \Return $t$\;
\end{algorithm}


\section{Advantages of Ehrenfeucht's Algorithm}\label{sec:advantages-of-ehrenfeucht's-algorithm}

\begin{itemize}
    \item \textbf{Efficiency:} The algorithm operates in $O(n^2)$ time, making it feasible for large graphs.
    \item \textbf{Simplicity:} The divide-and-conquer approach simplifies the process of identifying modules and constructing the decomposition tree.
    \item \textbf{Practicality:} The algorithm is applicable to both directed, undirected, and 2-structures graphs.
\end{itemize}


\hspace{4cm}

The next chapter will discuss an implementation of the algorithm in Python made by another Sorbonne University student and its performance compared with another implementation made in SageMath.

